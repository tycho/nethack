\documentstyle[titlepage]{article}

\textheight 220mm
\textwidth 160mm
\oddsidemargin 0mm
\evensidemargin 0mm
\topmargin 0mm

\newcommand{\nd}{\noindent}

\newcommand{\tb}[1]{\tt #1 \hfill}
\newcommand{\bb}[1]{\bf #1 \hfill}
\newcommand{\ib}[1]{\it #1 \hfill}

\newcommand{\blist}[1]
{\begin{list}{$\bullet$}
    {\leftmargin 30mm \topsep 2mm \partopsep 0mm \parsep 0mm \itemsep 1mm
     \labelwidth 28mm \labelsep 2mm
     #1}}

\newcommand{\elist}{\end{list}}

% this will make \tt underscores look better, but requires that
% math subscripts will never be used in this document
\catcode`\_=12

\begin{document}
%
% input file: guidebook.mn
% $Revision: 1.61.2.19 $ $Date: 2003/12/03 03:00:50 $
%
%.ds h0 "
%.ds h1 %.ds h2 \%
%.ds f0 "

%.mt
\title{\LARGE A Guide to the Mazes of Menace:\\
\Large Guidebook for {\it NetHack\/}}

%.au
\author{Eric S. Raymond\\
(Extensively edited and expanded for 3.4)}
\date{December 2, 2003}

\maketitle

%.hn 1
\section{Introduction}

%.pg

Recently, you have begun to find yourself unfulfilled and distant 
in your daily occupation.  Strange dreams of prospecting, stealing, 
crusading, and combat have haunted you in your sleep for many months, 
but you aren't sure of the reason.  You wonder whether you have in 
fact been having those dreams all your life, and somehow managed to 
forget about them until now.  Some nights you awaken suddenly
and cry out, terrified at the vivid recollection of the strange and 
powerful creatures that seem to be lurking behind every corner of the 
dungeon in your dream.  Could these details haunting your dreams be real?  
As each night passes, you feel the desire to enter the mysterious caverns 
near the ruins grow stronger.  Each morning, however, you quickly put 
the idea out of your head as you recall the tales of those who entered 
the caverns before you and did not return.  Eventually you can resist 
the yearning to seek out the fantastic place in your dreams no longer.  
After all, when other adventurers came back this way after spending time 
in the caverns, they usually seemed better off than when they passed 
through the first time.  And who was to say that all of those who did 
not return had not just kept going?
%.pg

Asking around, you hear about a bauble, called the Amulet of Yendor by some,
which, if you can find it, will bring you great wealth.  One legend you were
told even mentioned that the one who finds the amulet will be granted
immortality by the gods.  The amulet is rumored to be somewhere beyond the
Valley of Gehennom, deep within the Mazes of Menace.  Upon hearing the
legends, you immediately realize that there is some profound and 
undiscovered reason that you are to descend into the caverns and seek 
out that amulet of which they spoke.  Even if the rumors of the amulet's 
powers are untrue, you decide that you should at least be able to sell the 
tales of your adventures to the local minstrels for a tidy sum, especially 
if you encounter any of the terrifying and magical creatures of 
your dreams along the way.  You spend one last night fortifying yourself 
at the local inn, becoming more and more depressed as you watch the odds 
of your success being posted on the inn's walls getting lower and lower.  

%.pg
\nd In the morning you awake, collect your belongings, and 
set off for the dungeon.  After several days of uneventful 
travel, you see the ancient ruins that mark the entrance to the 
Mazes of Menace.  It is late at night, so you make camp at the entrance 
and spend the night sleeping under the open skies.  In the morning, you 
gather your gear, eat what may be your last meal outside, and enter the 
dungeon\ldots

%.hn 1
\section{What is going on here?}

%.pg
You have just begun a game of {\it NetHack}.  Your goal is to grab as much
treasure as you can, retrieve the Amulet of Yendor, and escape the
Mazes of Menace alive.  

%.pg
Your abilities and strengths for dealing with the hazards of adventure
will vary with your background and training:

%.pg
%
\blist{}
\item[\bb{Archeologists}]%
understand dungeons pretty well; this enables them
to move quickly and sneak up on the local nasties.  They start equipped
with the tools for a proper scientific expedition.
%.pg
%
\item[\bb{Barbarians}]%
are warriors out of the hinterland, hardened to battle.
They begin their quests with naught but uncommon strength, a trusty hauberk,
and a great two-handed sword.
%.pg
%
\item[\bb{Cavemen {\rm and} Cavewomen}]
start with exceptional strength, but unfortunately, neolithic weapons.
%.pg
%
\item[\bb{Healers}]%
are wise in medicine and apothecary.  They know the
herbs and simples that can restore vitality, ease pain, anesthetize,
and neutralize
poisons; and with their instruments, they can divine a being's state
of health or sickness.  Their medical practice earns them quite reasonable
amounts of money, with which they enter the dungeon.
%.pg
%
\item[\bb{Knights}]%
are distinguished from the common skirmisher by their
devotion to the ideals of chivalry and by the surpassing excellence of
their armor.
%.pg
%
\item[\bb{Monks}]%
are ascetics, who by rigorous practice of physical and mental
disciplines have become capable of fighting as effectively without weapons
as with.  They wear no armor but make up for it with increased mobility.
%.pg
%
\item[\bb{Priests {\rm and} Priestesses}]%
are clerics militant, crusaders
advancing the cause of righteousness with arms, armor, and arts
thaumaturgic.  Their ability to commune with deities via prayer
occasionally extricates them from peril, but can also put them in it.
%.pg
%
\item[\bb{Rangers}]%
are most at home in the woods, and some say slightly out
of place in a dungeon.  They are, however, experts in archery as well
as tracking and stealthy movement.
%.pg
%
\item[\bb{Rogues}]%
are agile and stealthy thieves, with knowledge of locks,
traps, and poisons.  Their advantage lies in surprise, which they employ
to great advantage.
%.pg
%
\item[\bb{Samurai}]%
are the elite warriors of feudal Nippon.  They are lightly
armored and quick, and wear the %
{\it dai-sho}, two swords of the deadliest
keenness.
%.pg
%
\item[\bb{Tourists}]%
start out with lots of gold (suitable for shopping with),
a credit card, lots of food, some maps, and an expensive camera.  Most
monsters don't like being photographed.
%.pg
%
\item[\bb{Valkyries}]%
are hardy warrior women.  Their upbringing in the harsh
Northlands makes them strong, inures them to extremes of cold, and instills
in them stealth and cunning.
%.pg
%
\item[\bb{Wizards}]%
start out with a knowledge of magic, a selection of magical
items, and a particular affinity for dweomercraft.  Although seemingly weak
and easy to overcome at first sight, an experienced Wizard is a deadly foe.
\elist

%.pg
You may also choose the race of your character:

%.pg
%
\blist{}
\item[\bb{Dwarves}]%
are smaller than humans or elves, but are stocky and solid
individuals.  Dwarves' most notable trait is their great expertise in mining
and metalwork.  Dwarvish armor is said to be second in quality not even to the
mithril armor of the Elves.
%.pg
%
\item[\bb{Elves}]%
are agile, quick, and perceptive; very little of what goes
on will escape an Elf.  The quality of Elven craftsmanship often gives
them an advantage in arms and armor.
%.pg
%
\item[\bb{Gnomes}]%
are smaller than but generally similar to dwarves.  Gnomes are
known to be expert miners, and it is known that a secret underground mine
complex built by this race exists within the Mazes of Menace, filled with
both riches and danger.
%.pg
%
\item[\bb{Humans}]%
are by far the most common race of the surface world, and
are thus the norm by which other races are often compared.  Although
they have no special abilities, they can succeed in any role.
%.pg
%
\item[\bb{Orcs}]%
are a cruel and barbaric race that hate every living thing
(including other orcs).  Above all others, Orcs hate Elves with a passion
unequalled, and will go out of their way to kill one at any opportunity.
The armor and weapons fashioned by the Orcs are typically of inferior quality.
\elist

%.hn 1
\section{What do all those things on the screen mean?}
%.pg
On the screen is kept a map of where you have been and what you have 
seen on the current dungeon level; as you explore more of the level, 
it appears on the screen in front of you.

%.pg
When {\it NetHack\/}'s ancestor {\it rogue\/} first appeared, its screen
orientation was almost unique among computer fantasy games.  Since
then, screen orientation has become the norm rather than the
exception; {\it NetHack\/} continues this fine tradition.  Unlike text
adventure games that accept commands in pseudo-English sentences and
explain the results in words, {\it NetHack\/} commands are all one or two
keystrokes and the results are displayed graphically on the screen.  A
minimum screen size of 24 lines by 80 columns is recommended; if the
screen is larger, only a $21\times80$ section will be used for the map.

%.pg
{\it NetHack\/} can even be played by blind players, with the assistance of
Braille readers or speech synthesisers.  Instructions for configuring
{\it NetHack\/} for the blind are included later in this document.

%.pg
{\it NetHack\/} generates a new dungeon every time you play it; even the
authors still find it an entertaining and exciting game despite
having won several times.

%.pg
{\it NetHack\/} offers a variety of display options.  The options available to
you will vary from port to port, depending on the capabilities of your
hardware and software, and whether various compile-time options were
enabled when your executable was created.  The three possible display
options are: a monochrome character interface, a color character interface,
and a graphical interface using small pictures called tiles.  The two
character interfaces allow fonts with other characters to be substituted,
but the default assignments use standard ASCII characters to represent
everything.  There is no difference between the various display options
with respect to game play.  Because we cannot reproduce the tiles or
colors in the Guidebook, and because it is common to all ports, we will
use the default ASCII characters from the monochrome character display
when referring to things you might see on the screen during your game.
%.pg
In order to understand what is going on in {\it NetHack}, first you must
understand what {\it NetHack\/} is doing with the screen.  The {\it NetHack\/}
screen replaces the ``You see \ldots'' descriptions of text adventure games.
Figure 1 is a sample of what a {\it NetHack\/} screen might look like.
The way the screen looks for you depends on your platform.

\vbox{
\begin{verbatim}
        The bat bites!

                ------
                |....|    ----------
                |.<..|####...@...$.|
                |....-#   |...B....+
                |....|    |.d......|
                ------    -------|--



        Player the Rambler     St:12 Dx:7 Co:18 In:11 Wi:9 Ch:15  Neutral
        Dlvl:1  $:0  HP:9(12) Pw:3(3) AC:10 Exp:1/19 T:257 Weak
\end{verbatim}
\begin{center}
Figure 1
\end{center}
}

%.hn 2
\subsection*{The status lines (bottom)}

%.pg
The bottom two lines of the screen contain several cryptic pieces of
information describing your current status.  If either status line
becomes longer than the width of the screen, you might not see all of
it.  Here are explanations of what the various status items mean
(though your configuration may not have all the status items listed
below):

%.lp
\blist{}
\item[\bb{Rank}]
Your character's name and professional ranking (based on the
experience level, see below).
%.lp
\item[\bb{Strength}]
A measure of your character's strength; one of your six basic
attributes.  A human character's attributes can range from 3 to 18 inclusive;
non-humans may exceed these limits
(occasionally you may get super-strengths of the form 18/xx, and magic can
also cause attributes to exceed the normal limits).  The
higher your strength, the stronger you are.  Strength affects how
successfully you perform physical tasks, how much damage you do in
combat, and how much loot you can carry.
%.lp
\item[\bb{Dexterity}]
Dexterity affects your chances to hit in combat, to avoid traps, and
do other tasks requiring agility or manipulation of objects.
%.lp
\item[\bb{Constitution}]
Constitution affects your ability to recover from injuries and other
strains on your stamina.
%.lp
\item[\bb{Intelligence}]
Intelligence affects your ability to cast spells and read spellbooks.
%.lp
\item[\bb{Wisdom}]
Wisdom comes from your practical experience (especially when dealing with
magic).  It affects your magical energy.
%.lp
\item[\bb{Charisma}]
Charisma affects how certain creatures react toward you.  In
particular, it can affect the prices shopkeepers offer you.
%.lp
\item[\bb{Alignment}]
%
{\it Lawful}, {\it Neutral\/} or {\it Chaotic}.  Often, Lawful is
taken as good and Chaotic is evil, but legal and ethical do not always
coincide.  Your alignment influences how other
monsters react toward you.  Monsters of a like alignment are more likely
to be non-aggressive, while those of an opposing alignment are more likely
to be seriously offended at your presence.
%.lp
\item[\bb{Dungeon Level}]
How deep you are in the dungeon.  You start at level one and the number
increases as you go deeper into the dungeon.  Some levels are special,
and are identified by a name and not a number.  The Amulet of Yendor is
reputed to be somewhere beneath the twentieth level.
%.lp
\item[\bb{Gold}]
The number of gold pieces you are openly carrying.  Gold which you have
concealed in containers is not counted.
%.lp
\item[\bb{Hit Points}]
Your current and maximum hit points.  Hit points indicate how much
damage you can take before you die.  The more you get hit in a fight,
the lower they get.  You can regain hit points by resting, or by using
certain magical items or spells.  The number in parentheses is the maximum
number your hit points can reach.
%.lp
\item[\bb{Power}]
Spell points.  This tells you how much mystic energy ({\it mana\/})
you have available for spell casting.  Again, resting will regenerate the
amount available.
%.lp
\item[\bb{Armor Class}]
A measure of how effectively your armor stops blows from unfriendly
creatures.  The lower this number is, the more effective the armor; it
is quite possible to have negative armor class.
%.lp
\item[\bb{Experience}]
Your current experience level and experience points.  As you
adventure, you gain experience points.  At certain experience point
totals, you gain an experience level.  The more experienced you are,
the better you fight and withstand magical attacks.  Many dungeons
show only your experience level here.
%.lp
\item[\bb{Time}]
The number of turns elapsed so far, displayed if you have the
{\it time\/} option set.
%.lp
\item[\bb{Hunger Status}]
Your current hunger status, ranging from %
{\it Satiated\/} down to {\it Fainting}.  If your hunger status is normal,
it is not displayed.
%.pg
Additional status flags may appear after the hunger status:
{\it Conf\/} when you're confused, {\it FoodPois\/} or {\it Ill\/}
when sick, {\it Blind\/}
when you can't see, {\it Stun\/} when stunned, and {\it Hallu\/} when
hallucinating.
\elist

%.hn 2
\subsection*{The message line (top)}

%.pg
The top line of the screen is reserved for messages that describe
things that are impossible to represent visually.  If you see a
``{\tt --More--}'' on the top line, this means that {\it NetHack\/} has
another message to display on the screen, but it wants to make certain
that you've read the one that is there first.  To read the next message,
just press the space bar.

%.hn 2
\subsection*{The map (rest of the screen)}

%.pg
The rest of the screen is the map of the level as you have explored it
so far.  Each symbol on the screen represents something.  You can set
various graphics
options to change some of the symbols the game uses; otherwise, the
game will use default symbols.  Here is a list of what the default
symbols mean:

\blist{}
%.lp
\item[\tb{- {\rm and} |}]
The walls of a room, or an open door.  Or a grave ({\tt |}).
%.lp
\item[\tb{.}]
The floor of a room, ice, or a doorless doorway.
%.lp
\item[\tb{\#}]
A corridor, or iron bars, or a tree, or possibly a kitchen sink (if
your dungeon has sinks), or a drawbridge.
%.lp
\item[\tb{>}]
Stairs down: a way to the next level.
%.lp
\item[\tb{<}]
Stairs up: a way to the previous level.
%.lp
\item[\tb{+}]
A closed door, or a spellbook containing a spell you may be able to learn.
%.lp
\item[\tb{@}]
Your character or a human.
%.lp
\item[\tb{\$}]
A pile of gold.
%.lp
\item[\tb{\^}]
A trap (once you have detected it).
%.lp
\item[\tb{)}]
A weapon.
%.lp
\item[\tb{[}]
A suit or piece of armor.
%.lp
\item[\tb{\%}]
Something edible (not necessarily healthy).
%.lp
\item[\tb{?}]
A scroll.
%.lp
\item[\tb{/}]
A wand.
%.lp
\item[\tb{=}]
A ring.
%.lp
\item[\tb{!}]
A potion.
%.lp
\item[\tb{(}]
A useful item (pick-axe, key, lamp \ldots).
%.lp
\item[\tb{"}]
An amulet or a spider web.
%.lp
\item[\tb{*}]
A gem or rock (possibly valuable, possibly worthless).
%.lp
\item[\tb{`}]
A boulder or statue.
%.lp
\item[\tb{0}]
An iron ball.
%.lp
\item[\tb{_}]
An altar, or an iron chain.
%.lp
\item[\tb{\{}]
A fountain.
%.lp
\item[\tb{\}}]
A pool of water or moat or a pool of lava.
%.lp
\item[\tb{$\backslash$}]
An opulent throne.
%.lp
\item[\tb{a-zA-Z {\rm \& other symbols}}]
Letters and certain other symbols represent the various inhabitants
of the Mazes of Menace.  Watch out, they can be nasty and vicious.
Sometimes, however, they can be helpful.
%.lp
\item[\tb{I}]
This marks the last known location of an invisible or otherwise unseen
monster.  Note that the monster could have moved.  The `F' and `m' commands
may be useful here.

\elist
%.pg
You need not memorize all these symbols; you can ask the game what any
symbol represents with the `{\tt /}' command (see the next section for
more info).

%.hn 1
\section{Commands}

%.pg
Commands are initiated by typing one or two characters.  Some commands,
like ``{\tt search}'', do not require that any more information be collected
by {\it NetHack\/}.  Other commands might require additional information, for
example a direction, or an object to be used.  For those commands that
require additional information, {\it NetHack\/} will present you with either 
a menu of choices, or with a command line prompt requesting information.  Which
you are presented with will depend chiefly on how you have set the
`{\it menustyle\/}'
option.

%.pg
For example, a common question in the form ``{\tt What do you want to
use? [a-zA-Z\ ?*]}'', asks you to choose an object you are carrying.
Here, ``{\tt a-zA-Z}'' are the inventory letters of your possible choices.
Typing `{\tt ?}' gives you an inventory list of these items, so you can see
what each letter refers to.  In this example, there is also a `{\tt *}'
indicating that you may choose an object not on the list, if you
wanted to use something unexpected.  Typing a `{\tt *}' lists your entire
inventory, so you can see the inventory letters of every object you're
carrying.  Finally, if you change your mind and decide you don't want
to do this command after all, you can press the `ESC' key to abort the
command.

%.pg
You can put a number before some commands to repeat them that many
times; for example, ``{\tt 10s}'' will search ten times.  If you have the
{\it number\_pad\/}
option set, you must type `{\tt n}' to prefix a count, so the example above
would be typed ``{\tt n10s}'' instead.  Commands for which counts make no
sense ignore them.  In addition, movement commands can be prefixed for
greater control (see below).  To cancel a count or a prefix, press the
`ESC' key.

%.pg
The list of commands is rather long, but it can be read at any time
during the game through the `{\tt ?}' command, which accesses a menu of
helpful texts.  Here are the commands for your reference:

\blist{}
%.lp
\item[\tb{?}]
Help menu:  display one of several help texts available.
%.lp
\item[\tb{/}]
Tell what a symbol represents.  You may choose to specify a location
or type a symbol (or even a whole word) to explain.
Specifying a location is done by moving the cursor to a particular spot
on the map and then pressing one of `{\tt .}', `{\tt ,}', `{\tt ;}',
or `{\tt :}'.  `{\tt .}' will explain the symbol at the chosen location,
conditionally check for ``{\tt More info?}'' depending upon whether the
{\it help\/}
option is on, and then you will be asked to pick another location;
`{\tt ,}' will explain the symbol but skip any additional
information; `{\tt ;}' will skip additional info and also not bother asking
you to choose another location to examine; `{\tt :}' will show additional
info, if any, without asking for confirmation.  When picking a location,
pressing the {\tt ESC} key will terminate this command, or pressing `{\tt ?}'
will give a brief reminder about how it works.

%.pg
Specifying a name rather than a location
always gives any additional information available about that name.
%.lp
\item[\tb{\&}]
Tell what a command does.
%.lp
\item[\tb{<}]
Go up to the previous level (if you are on a staircase or ladder).
%.lp
\item[\tb{>}]
Go down to the next level (if you are on a staircase or ladder).
%.lp
\item[\tb{[yuhjklbn]}]
Go one step in the direction indicated (see Figure 2).  If you sense
or remember
a monster there, you will fight the monster instead.  Only these
one-step movement commands cause you to fight monsters; the others
(below) are ``safe.''
%.sd
\begin{center}
\begin{tabular}{cc}
\verb+   y  k  u   + & \verb+   7  8  9   +\\
\verb+    \ | /    + & \verb+    \ | /    +\\
\verb+   h- . -l   + & \verb+   4- . -6   +\\
\verb+    / | \    + & \verb+    / | \    +\\
\verb+   b  j  n   + & \verb+   1  2  3   +\\
                     & (if {\it number\_pad\/} set)
\end{tabular}
\end{center}
%.ed
\begin{center}
Figure 2
\end{center}
%.lp
\item[\tb{[YUHJKLBN]}]
Go in that direction until you hit a wall or run into something.
%.lp
\item[\tb{m[yuhjklbn]}]
Prefix:  move without picking up objects or fighting (even if you remember
a monster there)
%.lp
\item[\tb{F[yuhjklbn]}]
Prefix:  fight a monster (even if you only guess one is there)
%.lp
\item[\tb{M[yuhjklbn]}]
Prefix:  Move far, no pickup.
%.lp
\item[\tb{g[yuhjklbn]}]
Prefix:  Move until something interesting is found.
%.lp
\item[\tb{G[yuhjklbn] {\rm or} <CONTROL->[yuhjklbn]}]
Prefix:  Same as `{\tt g}', but forking of corridors is not considered
interesting.
%.lp
\item[\tb{_}]
Travel to a map location via a shortest-path algorithm.  The shortest path
is computed over map locations the hero knows about (e.g. seen or
previously traversed).  If there is no known path, a guess is made instead.
Stops on most of 
the same conditions as the `G' command, but without picking up
objects, similar to the `M' command.  For ports with mouse 
support, the command is also invoked when a mouse-click takes place on a 
location other than the current position.
%.lp
\item[\tb{.}]
Rest, do nothing for one turn.
%.lp
\item[\tb{a}]
Apply (use) a tool (pick-axe, key, lamp \ldots).
%.lp
\item[\tb{A}]
Remove one or more worn items, such as armor.
Use `{\tt T}' (take off) to take off only one piece of armor 
or `{\tt R}' (remove) to take off only one accessory.
%.lp
\item[\tb{\^{}A}]
Redo the previous command.
%.lp
\item[\tb{c}]
Close a door.
%.lp
\item[\tb{C}]
Call (name) an individual monster.
%.lp
\item[\tb{\^{}C}]
Panic button.  Quit the game.
%.lp
\item[\tb{d}]
Drop something.\\
{\tt d7a} --- drop seven items of object
{\it a}.
%.lp
\item[\tb{D}]
Drop several things.  In answer to the question
``{\tt What kinds of things do you want to drop? [!\%= BUCXaium]}''
you should type zero or more object symbols possibly followed by
`{\tt a}' and/or `{\tt i}' and/or `{\tt u}' and/or `{\tt m}'.
In addition, one or more of
the blessed/uncursed/cursed groups may be typed.\\
%.sd
%.si
{\tt DB}  --- drop all objects known to be blessed.\\
{\tt DU}  --- drop all objects known to be uncursed.\\
{\tt DC}  --- drop all objects known to be cursed.\\
{\tt DX}  --- drop all objects of unknown B/U/C status.\\
{\tt Da}  --- drop all objects, without asking for confirmation.\\
{\tt Di}  --- examine your inventory before dropping anything.\\
{\tt Du}  --- drop only unpaid objects (when in a shop).\\
{\tt Dm}  --- use a menu to pick which object(s) to drop.\\
{\tt D\%u} --- drop only unpaid food.
%.ei
%.ed
%.lp
\item[\tb{\^{}D}]
Kick something (usually a door).
%.lp
\item[\tb{e}]
Eat food.
%.lp
% Make sure Elbereth is not hyphenated below, the exact spelling matters.
% (Only specified here to parallel Guidebook.mn; use of \tt font implicity
% prevents automatic hyphenation in TeX and LaTeX.)
\hyphenation{Elbereth}		%override the deduced syllable breaks
\item[\tb{E}]
Engrave a message on the floor.
Engraving the word ``{\tt Elbereth}'' will cause most monsters to not attack
you hand-to-hand (but if you attack, you will rub it out); this is
often useful to give yourself a breather.  (This feature may be compiled out
of the game, so your version might not have it.)\\
%.sd
%.si
{\tt E-} --- write in the dust with your fingers.
%.ei
%.ed
%.Ip
\item[\tb{f}]
Fire one of the objects placed in your quiver.  You may select
ammunition with a previous `{\tt Q}' command, or let the computer pick
something appropriate if {\it autoquiver\/} is true.
%.lp
\item[\tb{i}]
List your inventory (everything you're carrying).
%.lp
\item[\tb{I}]
List selected parts of your inventory.\\
%.sd
%.si
{\tt I*} --- list all gems in inventory;\\
{\tt Iu} --- list all unpaid items;\\
{\tt Ix} --- list all used up items that are on your shopping bill;\\
{\tt I\$} --- count your money.
%.ei
%.ed
%.lp
\item[\tb{o}]
Open a door.
%.lp
\item[\tb{O}]
Set options.  A menu showing the current option values will be
displayed.  You can change most values simply by selecting the menu
entry for the given option (ie, by typing its letter or clicking upon
it, depending on your user interface).  For the non-boolean choices,
a further menu or prompt will appear once you've closed this menu.
The available options
are listed later in this Guidebook.  Options are usually set before the
game rather than with the `{\tt O}' command; see the section on options below.
%.lp
\item[\tb{p}]
Pay your shopping bill.
%.lp
\item[\tb{P}]
Put on a ring or other accessory (amulet, blindfold).
%.lp
\item[\tb{\^{}P}]
Repeat previous message.  Subsequent {\tt \^{}P}'s repeat earlier messages.
The behavior can be varied via the msg_window option.
%.lp
\item[\tb{q}]
Quaff (drink) something (potion, water, etc).
%.lp
\item[\tb{Q}]
Select an object for your quiver.  You can then throw this using
the `f' command.  (In versions prior to 3.3 this was the command to quit
the game, which has now been moved to `{\tt \#quit}'.)
%.lp
\item[\tb{r}]
Read a scroll or spellbook.
%.lp
\item[\tb{R}]
Remove an accessory (ring, amulet, etc).
%.lp
\item[\tb{\^{}R}]
Redraw the screen.
%.lp
\item[\tb{s}]
Search for secret doors and traps around you.  It usually takes several
tries to find something.
%.lp
\item[\tb{S}]
Save (and suspend) the game.  The game will be restored automatically the
next time you play.
%.lp
\item[\tb{t}]
Throw an object or shoot a projectile.
%.lp
\item[\tb{T}]
Take off armor.
%.lp
\item[\tb{\^{}T}]
Teleport, if you have the ability.
%.lp
\item[\tb{v}]
Display version number.
%.lp
\item[\tb{V}]
Display the game history.
%.lp
\item[\tb{w}]
Wield weapon.\\
%.sd
%.si
{\tt w-} --- wield nothing, use your bare hands.
%.ei
%.ed
%.lp
\item[\tb{W}]
Wear armor.
%.lp
\item[\tb{x}]
Exchange your wielded weapon with the item in your alternate
weapon slot.  The latter is used as your secondary weapon when engaging in
two-weapon combat.  Note that if one of these slots is empty,
the exchange still takes place.
%.lp
\item[\tb{X}]
Enter explore (discovery) mode, explained in its own section later.
%.lp
\item[\tb{\^{}X}]
Display your name, role, race, gender, and alignment as well as
the various deities in your game.
%.lp
\item[\tb{z}]
Zap a wand.  To aim at yourself, use `{\tt .}' for the direction.
%.lp
\item[\tb{Z}]
Zap (cast) a spell.  To cast at yourself, use `{\tt .}' for the direction.
%.lp
\item[\tb{\^{}Z}]
Suspend the game (UNIX versions with job control only).
%.lp
\item[\tb{:}]
Look at what is here.
%.lp
\item[\tb{;}]
Show what type of thing a visible symbol corresponds to.
%.lp
\item[\tb{,}]
Pick up some things. May be preceded by `{\tt m}' to force a selection menu.
%.lp
\item[\tb{@}]
Toggle the {\it autopickup\/} option on and off.
%.lp
\item[\tb{\^{}}]
Ask for the type of a trap you found earlier.
%.lp
\item[\tb{)}]
Tell what weapon you are wielding.
%.lp
\item[\tb{[}]
Tell what armor you are wearing.
%.lp
\item[\tb{=}]
Tell what rings you are wearing.
%.lp
\item[\tb{"}]
Tell what amulet you are wearing.
%.lp
\item[\tb{(}]
Tell what tools you are using.
%.lp
\item[\tb{*}]
Tell what equipment you are using; combines the preceding five type-specific
commands into one.
%.lp
\item[\tb{\$}]
Count your gold pieces.
%.lp
\item[\tb{+}]
List the spells you know.  Using this command, you can also rearrange
the order in which your spells are listed.  They are shown via a menu,
and if you select a spell in that menu, you'll be re-prompted for
another spell to swap places with it, and then have opportunity to
make further exchanges.
%.lp
\item[\tb{$\backslash$}]
Show what types of objects have been discovered.
%.lp
\item[\tb{!}]
Escape to a shell.
%.lp
\item[\tb{\#}]
Perform an extended command.  As you can see, the authors of {\it NetHack\/}
used up all the letters, so this is a way to introduce the less frequently
used commands.
What extended commands are available depends on what features
the game was compiled with.
%.lp
\item[\tb{\#adjust}]
Adjust inventory letters (most useful when the
{\it fixinv\/}
option is ``on'').
%.lp
\item[\tb{\#chat}]
Talk to someone.
%.lp
\item[\tb{\#conduct}]
List which challenges you have adhered to.  See the section below entitled
``Conduct'' for details.
%.lp
\item[\tb{\#dip}]
Dip an object into something.
%.lp
\item[\tb{\#enhance}]
Advance or check weapons and spell skills.
%.lp
\item[\tb{\#force}]
Force a lock.
%.lp
\item[\tb{\#invoke}]
Invoke an object's special powers.
%.lp
\item[\tb{\#jump}]
Jump to another location.
%.lp
\item[\tb{\#loot}]
Loot a box or bag on the floor beneath you, or the saddle 
from a horse standing next to you.
%.lp
\item[\tb{\#monster}]
Use a monster's special ability (when polymorphed into monster form).
%.lp
\item[\tb{\#name}]
Name an item or type of object.
%.lp
\item[\tb{\#offer}]
Offer a sacrifice to the gods.
%.lp
\item[\tb{\#pray}]
Pray to the gods for help.
%.lp
\item[\tb{\#quit}]
Quit the program without saving your game.
%.lp
\item[\tb{\#ride}]
Ride (or stop riding) a monster.
%.lp
\item[\tb{\#rub}]
Rub a lamp or a stone.
%.lp
\item[\tb{\#sit}]
Sit down.
%.lp
\item[\tb{\#turn}]
Turn undead.
%.lp
\item[\tb{\#twoweapon}]
Toggle two-weapon combat on or off.  Note that you must
use suitable weapons for this type of combat, or it will
be automatically turned off.
%.lp
\item[\tb{\#untrap}]
Untrap something (trap, door, or chest).
%.lp
\item[\tb{\#version}]
Print compile time options for this version of {\it NetHack}.
%.lp
\item[\tb{\#wipe}]
Wipe off your face.
%.lp
\item[\tb{\#?}]
Help menu:  get the list of available extended commands.
\elist

%.pg
\nd If your keyboard has a meta key (which, when pressed in combination
with another key, modifies it by setting the `meta' [8th, or `high']
bit), you can invoke many extended commands by meta-ing the first
letter of the command.
%- In {\it NT, OS/2, PC\/ {\rm and} ST NetHack},
%- the `Alt' key can be used in this fashion;
%- on the Amiga set the {\it altmeta\/} option to get this behavior.
In {\it NT, OS/2, {\rm and} PC NetHack},
the `Alt' key can be used in this fashion.
\blist{}
%.lp 
\item[\tb{M-?}]
{\tt\#?} (not supported by all platforms)
%.lp
\item[\tb{M-2}]
{\tt\#twoweapon} (unless the {\it number\_pad\/} option is enabled)
%.lp
\item[\tb{M-a}]
{\tt\#adjust}
%.lp
\item[\tb{M-c}]
{\tt\#chat}
%.lp
\item[\tb{M-d}]
{\tt\#dip}
%.lp
\item[\tb{M-e}]
{\tt\#enhance}
%.lp
\item[\tb{M-f}]
{\tt\#force}
%.lp
\item[\tb{M-i}]
{\tt\#invoke}
%.lp
\item[\tb{M-j}]
{\tt\#jump}
%.lp
\item[\tb{M-l}]
{\tt\#loot}
%.lp
\item[\tb{M-m}]
{\tt\#monster}
%.lp
\item[\tb{M-n}]
{\tt\#name}
%.lp
\item[\tb{M-o}]
{\tt\#offer}
%.lp
\item[\tb{M-p}]
{\tt\#pray}
%.Ip
\item[\tb{M-q}]
{\tt\#quit}
%.lp
\item[\tb{M-r}]
{\tt\#rub}
%.lp
\item[\tb{M-s}]
{\tt\#sit}
%.lp
\item[\tb{M-t}]
{\tt\#turn}
%.lp
\item[\tb{M-u}]
{\tt\#untrap}
%.lp
\item[\tb{M-v}]
{\tt\#version}
%.lp
\item[\tb{M-w}]
{\tt\#wipe}
\elist

%.pg
\nd If the {\it number\_pad\/} option is on, some additional letter commands
are available:
\blist{}
%.lp 
\item[\tb{h}]
Help menu:  display one of several help texts available, like ``{\tt ?}''.
%.lp
\item[\tb{j}]
Jump to another location.  Same as ``{\tt \#jump}'' or ``{\tt M-j}''.
%.lp
\item[\tb{k}]
Kick something (usually a door).  Same as `{\tt \^{}D}'.
%.lp
\item[\tb{l}]
Loot a box or bag on the floor beneath you, or the saddle 
from a horse standing next to you.  Same as ``{\tt \#loot}'' or ``{\tt M-l}''.
%.lp
\item[\tb{N}]
Name an object or type of object.  Same as ``{\tt \#name}'' or ``{\tt M-n}''.
%.lp
\item[\tb{u}]
Untrap a trap, door, or chest.  Same as ``{\tt \#untrap}'' or ``{\tt M-u}''.
\elist

%.hn 1
\section{Rooms and corridors}

%.pg
Rooms and corridors in the dungeon are either lit or dark.
Any lit areas within your line of sight will be displayed;
dark areas are only displayed if they are within one space of you. 
Walls and corridors remain on the map as you explore them.

%.pg
Secret corridors are hidden.  You can find them with the `{\tt s}' (search)
command.

%.hn 2
\subsection*{Doorways}

%.pg
Doorways connect rooms and corridors.  Some doorways have no doors;
you can walk right through.  Others have doors in them, which may be
open, closed, or locked.  To open a closed door, use the `{\tt o}' (open)
command; to close it again, use the `{\tt c}' (close) command.

%.pg
You can get through a locked door by using a tool to pick the lock
with the `{\tt a}' (apply) command, or by kicking it open with the
`{\tt \^{}D}' (kick) command.

%.pg
Open doors cannot be entered diagonally; you must approach them
straight on, horizontally or vertically.  Doorways without doors are
not restricted in this fashion.

%.pg
Doors can be useful for shutting out monsters.  Most monsters cannot
open doors, although a few don't need to (ex.\ ghosts can walk through
doors).

%.pg
Secret doors are hidden.  You can find them with the `{\tt s}' (search)
command.  Once found they are in all ways equivalent to normal doors.

%.hn 2
\subsection*{Traps (`{\tt \^{}}')}

%.pg
There are traps throughout the dungeon to snare the unwary delver.
For example, you may suddenly fall into a pit and be stuck for a few
turns trying to climb out.  Traps don't appear on your map until you
see one triggered by moving onto it, see something fall into it, or you
discover it with the `{\tt s}' (search) command.  Monsters can fall prey to
traps, too, which can be a very useful defensive strategy.

%.pg
There is a special pre-mapped branch of the dungeon based on the
classic computer game ``{\tt Sokoban}.''  The goal is to push the boulders
into the pits or holes.  With careful foresight, it is possible to
complete all of the levels according to the traditional rules of
Sokoban.  Some allowances are permitted in case the player gets stuck;
however, they will lower your luck.

\subsection*{Stairs (`{\tt <}', `{\tt >}')}

%.pg
In general, each level in the dungeon will have a staircase going up
(`{\tt <}') to the previous level and another going down (`{\tt >}')
to the next
level.  There are some exceptions though.  For instance, fairly early
in the dungeon you will find a level with two down staircases, one
continuing into the dungeon and the other branching into an area
known as the Gnomish Mines.  Those mines eventually hit a dead end,
so after exploring them (if you choose to do so), you'll need to
climb back up to the main dungeon.

%.pg
When you traverse a set of stairs, or trigger a trap which sends you
to another level, the level you're leaving will be deactivated and
stored in a file on disk.  If you're moving to a previously visited
level, it will be loaded from its file on disk and reactivated.  If
you're moving to a level which has not yet been visited, it will be
created (from scratch for most random levels, from a template for
some ``special'' levels, or loaded from the remains of an earlier game
for a ``bones'' level as briefly described below).  Monsters are only
active on the current level; those on other levels are essentially
placed into stasis.
 
%.pg
Ordinarily when you climb a set of stairs, you will arrive on the
corresponding staircase at your destination.  However, pets (see below)
and some other monsters will follow along if they're close enough when
you travel up or down stairs, and occasionally one of these creatures
will displace you during the climb.  When that occurs, the pet or other
monster will arrive on the staircase and you will end up nearby.

\subsection*{Ladders (`{\tt <}', `{\tt >}')}

%.pg
Ladders serve the same purpose as staircases, and the two types of
inter-level connections are nearly indistinguishable during game play.

%.hn 2
\subsection*{Shops and shopping}

%.pg
Occasionally you will run across a room with a shopkeeper near the door
and many items lying on the floor.  You can buy items by picking them
up and then using the `{\tt p}' command.  You can inquire about the price
of an item prior to picking it up by using the ``{\tt \#chat}'' command
while standing on it.  Using an item prior to paying for it will incur a
charge, and the shopkeeper won't allow you to leave the shop until you
have paid any debt you owe.

%.pg
You can sell items to a shopkeeper by dropping them to the floor while
inside a shop.  You will either be offered an amount of gold and asked
whether you're willing to sell, or you'll be told that the shopkeeper
isn't interested (generally, your item needs to be compatible with the
type of merchandise carried by the shop).

%.pg
If you drop something in a shop by accident, the shopkeeper will usually
claim ownership without offering any compensation.  You'll have to buy
it back if you want to reclaim it.

%.pg
Shopkeepers sometimes run out of money.  When that happens, you'll be
offered credit instead of gold when you try to sell something.  Credit
can be used to pay for purchases, but it is only good in the shop where
it was obtained; other shopkeepers won't honor it.  (If you happen to
find a ``credit card'' in the dungeon, don't bother trying to use it in
shops; shopkeepers will not accept it.)

%.pg
The {\tt \$} command, which reports the amount of gold you are carrying
(in inventory, not inside bags or boxes), will also show current shop
debt or credit, if any.  The {\tt Iu} command lists unpaid items
(those which still belong to the shop) if you are carrying any.
The {\tt Ix} command shows an inventory-like display of any unpaid
items which have been used up, along with other shop fees, if any.

%.hn 3
\subsubsection*{Shop idiosyncracies}

%.pg
Several aspects of shop behavior might be unexpected.

\begin{itemize}
% note: a bullet is the default item label so we could omit [$\bullet$] here
%.lp \(bu 2
\item[$\bullet$]
The price of a given item can vary due to a variety of factors.
%.lp \(bu 2
\item[$\bullet$]
A shopkeeper treats the spot immediately inside the door as if it were
outside the shop.
%.lp \(bu 2
\item[$\bullet$]
While the shopkeeper watches you like a hawk, he will generally ignore
any other customers.
%.lp \(bu 2
\item[$\bullet$]
If a shop is ``closed for inventory'', it will not open of its own accord.
%.lp \(bu 2
\item[$\bullet$]
Shops do not get restocked with new items, regardless of inventory depletion.
\end{itemize}

%.hn 1
\section{Monsters}

%.pg
Monsters you cannot see are not displayed on the screen.  Beware!
You may suddenly come upon one in a dark place.  Some magic items can
help you locate them before they locate you (which some monsters can do
very well).

%.pg
The commands `{\tt /}' and `{\tt ;}' may be used to obtain information
about those
monsters who are displayed on the screen.  The command `{\tt C}' allows you
to assign a name to a monster, which may be useful to help distinguish
one from another when multiple monsters are present.  Assigning a name
which is just a space will remove any prior name.

%.pg
The extended command ``{\tt \#chat}'' can be used to interact with an adjacent
monster.  There is no actual dialog (in other words, you don't get to
choose what you'll say), but chatting with some monsters such as a
shopkeeper or the Oracle of Delphi can produce useful results.

%.hn 2
\subsection*{Fighting}

%.pg
If you see a monster and you wish to fight it, just attempt to walk
into it.  Many monsters you find will mind their own business unless
you attack them.  Some of them are very dangerous when angered.
Remember:  discretion is the better part of valor.

%.pg
If you can't see a monster (if it is invisible, or if you are blinded),
the symbol `I' will be shown when you learn of its presence.
If you attempt to walk into it, you will try to fight it just like
a monster that you can see; of course,
if the monster has moved, you will attack empty air.  If you guess
that the monster has moved and you don't wish to fight, you can use the `m'
command to move without fighting; likewise, if you don't remember a monster
but want to try fighting anyway, you can use the `F' command.

%.hn 2
\subsection*{Your pet}

%.pg
You start the game with a little dog (`{\tt d}'), cat (`{\tt f}'),
or pony (`{\tt u}'), which follows
you about the dungeon and fights monsters with you.  Like you, your
pet needs food to survive.  It usually feeds itself on fresh carrion
and other meats.  If you're worried about it or want to train it, you
can feed it, too, by throwing it food.  A properly trained pet can be
very useful under certain circumstances.

%.pg
Your pet also gains experience from killing monsters, and can grow
over time, gaining hit points and doing more damage.  Initially, your
pet may even be better at killing things than you, which makes pets
useful for low-level characters.

%.pg
Your pet will follow you up and down staircases if it is next to you
when you move.  Otherwise your pet will be stranded and may become
wild.  Similarly, when you trigger certain types of traps which alter
your location (for instance, a trap door which drops you to a lower
dungeon level), any adjacent pet will accompany you and any non-adjacent
pet will be left behind.  Your pet may trigger such traps itself; you
will not be carried along with it even if adjacent at the time.

%.hn 2
\subsection*{Steeds}

%.pg
Some types of creatures in the dungeon can actually be ridden if you
have the right equipment and skill.  Convincing a wild beast to let
you saddle it up is difficult to say the least.  Many a dungeoneer
has had to resort to magic and wizardry in order to forge the alliance.
Once you do have the beast under your control however, you can
easily climb in and out of the saddle with the `{\tt \#ride}' command.  Lead
the beast around the dungeon when riding, in the same manner as
you would move yourself.  It is the beast that you will see displayed
on the map.

%.pg
Riding skill is managed by the `{\tt \#enhance}' command.  See the section
on Weapon proficiency for more information about that.

%.hn 2
\subsection*{Bones levels}

%.pg
You may encounter the shades and corpses of other adventurers (or even
former incarnations of yourself!) and their personal effects.  Ghosts
are hard to kill, but easy to avoid, since they're slow and do little
damage.  You can plunder the deceased adventurer's possessions;
however, they are likely to be cursed.  Beware of whatever killed the
former player; it is probably still lurking around, gloating over its
last victory.

%.hn 1
\section{Objects}

%.pg
When you find something in the dungeon, it is common to want to pick
it up.  In {\it NetHack}, this is accomplished automatically by walking over
the object (unless you turn off the {\it autopickup\/}
option (see below), or move with the `{\tt m}' prefix (see above)), or
manually by using the `{\tt ,}' command.
%.pg
If you're carrying too many items, {\it NetHack\/} will tell you so and you
won't be able to pick up anything more.  Otherwise, it will add the object(s)
to your pack and tell you what you just picked up.
%.pg
As you add items to your inventory, you also add the weight of that object
to your load.  The amount that you can carry depends on your strength and
your constitution.  The
stronger you are, the less the additional load will affect you.  There comes
a point, though, when the weight of all of that stuff you are carrying around
with you through the dungeon will encumber you.  Your reactions
will get slower and you'll burn calories faster, requiring food more frequently
to cope with it.  Eventually, you'll be so overloaded that you'll either have
to discard some of what you're carrying or collapse under its weight.
%.pg
NetHack will tell you how badly you have loaded yourself.  The symbols
`Burdened', `Stressed', `Strained', `Overtaxed' and `Overloaded' are
displayed on the bottom line display to indicate your condition.

%.pg
When you pick up an object, it is assigned an inventory letter.  Many
commands that operate on objects must ask you to find out which object
you want to use.  When {\it NetHack\/} asks you to choose a particular object
you are carrying, you are usually presented with a list of inventory
letters to choose from (see Commands, above).

%.pg
Some objects, such as weapons, are easily differentiated.  Others, like
scrolls and potions, are given descriptions which vary according to
type.  During a game, any two objects with the same description are
the same type.  However, the descriptions will vary from game to game.

%.pg
When you use one of these objects, if its effect is obvious, {\it NetHack\/}
will remember what it is for you.  If its effect isn't extremely
obvious, you will be asked what you want to call this type of object
so you will recognize it later.  You can also use the ``{\tt \#name}''
command for the same purpose at any time, to name all objects of a
particular type or just an individual object.
When you use ``{\tt \#name}'' on an object which has already been named,
specifying a space as the value will remove the prior name instead
of assigning a new one.

%.hn 2
\subsection*{Curses and Blessings}

%.pg
Any object that you find may be cursed, even if the object is
otherwise helpful.  The most common effect of a curse is being stuck
with (and to) the item.  Cursed weapons weld themselves to your hand
when wielded, so you cannot unwield them.  Any cursed item you wear
is not removable by ordinary means.  In addition, cursed arms and armor
usually, but not always, bear negative enchantments that make them
less effective in combat.  Other cursed objects may act poorly or
detrimentally in other ways.

%.pg
Objects can also be blessed.  Blessed items usually work better or
more beneficially than normal uncursed items.  For example, a blessed
weapon will do more damage against demons.

%.pg
There are magical means of bestowing or removing curses upon objects,
so even if you are stuck with one, you can still have the curse
lifted and the item removed.  Priests and Priestesses have an innate
sensitivity to this property in any object, so they can more easily avoid
cursed objects than other character roles.

%.pg
An item with unknown status will be reported in your inventory with no prefix.
An item which you know the state of will be distinguished in your inventory
by the presence of the word ``cursed'', ``uncursed'' or ``blessed'' in the
description of the item.

%.hn 2
\subsection*{Weapons (`{\tt )}')}

%.pg
Given a chance, most monsters in the Mazes of Menace will gratuitously try to
kill you.  You need weapons for self-defense (killing them first).  Without a
weapon, you do only 1--2 hit points of damage (plus bonuses, if any).
Monk characters are an exception; they normally do much more damage with
bare hands than they do with weapons.

%.pg
There are wielded weapons, like maces and swords, and thrown weapons,
like arrows and spears.  To hit monsters with a weapon, you must wield it and
attack them, or throw it at them.  You can simply elect to throw a spear.
To shoot an arrow, you should first wield a bow, then throw the arrow.
Crossbows shoot crossbow bolts.  Slings hurl rocks and (other) stones
(like gems).

%.pg
Enchanted weapons have a ``plus'' (or ``to hit enhancement'' which can be
either positive or negative) that adds to your chance to
hit and the damage you do to a monster.  The only way to determine a weapon's
enchantment is to have it magically identified somehow.
Most weapons are subject to some type of damage like rust.  Such
``erosion'' damage can be repaired.

%.pg
The chance that an attack will successfully hit a monster, and the amount
of damage such a hit will do, depends upon many factors.  Among them are:
type of weapon, quality of weapon (enchantment and/or erosion), experience
level, strength, dexterity, encumbrance, and proficiency (see below).  The
monster's armor class---a general defense rating, not necessarily due to
wearing of armor---is a factor too; also, some monsters are particularly
vulnerable to certain types of weapons.

%.pg
Many weapons can be wielded in one hand; some require both hands.
When wielding a two-handed weapon, you can not wear a shield, and
vice versa.  When wielding a one-handed weapon, you can have another
weapon ready to use by setting things up with the `{\tt x}' command, which
exchanges your primary (the one being wielded) and alternate weapons.
And if you have proficiency in the ``two weapon combat'' skill, you
may wield both weapons simultaneously as primary and secondary; use the
`{\tt \#twoweapon}' extended command to engage or disengage that.  Only
some types of characters (barbarians, for instance) have the necessary
skill available.  Even with that skill, using two weapons at once incurs
a penalty in the chance to hit your target compared to using just one
weapon at a time.

%.pg
There might be times when you'd rather not wield any weapon at all.
To accomplish that, wield `{\tt -}', or else use the `{\tt A}' command which
allows you to unwield the current weapon in addition to taking off
other worn items.

%.pg
Those of you in the audience who are AD\&D players, be aware that each
weapon which existed in AD\&D does roughly the same damage to monsters in
{\it NetHack}.  Some of the more obscure weapons (such as the %
{\it aklys}, {\it lucern hammer}, and {\it bec-de-corbin\/}) are defined
in an appendix to {\it Unearthed Arcana}, an AD\&D supplement.

%.pg
The commands to use weapons are `{\tt w}' (wield), `{\tt t}' (throw),
`{\tt f}' (fire, an alternative way of throwing), `{\tt Q}' (quiver),
`{\tt x}' (exchange), `{\tt \#twoweapon}', and `{\tt \#enhance}' (see below).

%.hn 3
\subsection*{Throwing and shooting}

%.pg
You can throw just about anything via the `{\tt t}' command.  It will prompt
for the item to throw; picking `{\tt ?}' will list things in your inventory
which are considered likely to be thrown, or picking `{\tt *}' will list
your entire inventory.  After you've chosen what to throw, you will
be prompted for a direction rather than for a specific target.  The
distance something can be thrown depends mainly on the type of object
and your strength.  Arrows can be thrown by hand, but can be thrown
much farther and will be more likely to hit when thrown while you are
wielding a bow.

%.pg
You can simplify the throwing operation by using the `{\tt Q}' command to
select your preferred ``missile'', then using the `{\tt f}' command to
throw it.  You'll be prompted for a direction as above, but you don't
have to specify which item to throw each time you use `{\tt f}'.  There is
also an option,
{\it autoquiver},
which has {\it NetHack\/} choose another item to automatically fill your
quiver when the inventory slot used for `{\tt Q}' runs out.

%.pg
Some characters have the ability to fire a volley of multiple items in a
single turn.  Knowing how to load several rounds of ammunition at
once---or hold several missiles in your hand---and still hit a
target is not an easy task.  Rangers are among those who are adept
at this task, as are those with a high level of proficiency in the
relevant weapon skill (in bow skill if you're wielding one to
shoot arrows, in crossbow skill if you're wielding one to shoot bolts,
or in sling skill if you're wielding one to shoot stones).
The number of items that the character has a chance to fire varies from
turn to turn.  You can explicitly limit the number of shots by using a
numeric prefix before the `{\tt t}' or `{\tt f}' command.
For example, ``{\tt 2f}'' (or ``{\tt n2f}'' if using
{\it number\_pad\/}
mode) would ensure that at most 2 arrows are shot
even if you could have fired 3.  If you specify
a larger number than would have been shot (``{\tt 4f}'' in this example),
you'll just end up shooting the same number (3, here) as if no limit
had been specified.  Once the volley is in motion, all of the items
will travel in the same direction; if the first ones kill a monster,
the others can still continue beyond that spot.

%.hn 3
\subsection*{Weapon proficiency}

%.pg
You will have varying degrees of skill in the weapons available.
Weapon proficiency, or weapon skills, affect how well you can use
particular types of weapons, and you'll be able to improve your skills
as you progress through a game, depending on your role, your experience
level, and use of the weapons.

%.pg
For the purposes of proficiency, weapons have
been divided up into various groups such as daggers, broadswords, and
polearms.  Each role has a limit on what level of proficiency a character
can achieve for each group.  For instance, wizards can become highly
skilled in daggers or staves but not in swords or bows.

%.pg
The `{\tt \#enhance}' extended command is used to review current weapons proficiency
(also spell proficiency) and to choose which skill(s) to improve when
you've used one or more skills enough to become eligible to do so.  The
skill rankings are ``none'' (sometimes also referred to as ``restricted'',
because you won't be able to advance), ``unskilled'', ``basic'', ``skilled'',
and ``expert''.  Restricted skills simply will not appear in the list
shown by `{\tt \#enhance}'.  (Divine intervention might unrestrict a particular
skill, in which case it will start at unskilled and be limited to basic.)
Some characters can enhance their barehanded combat or martial arts skill
beyond expert to ``master'' or ``grand master''.

%.pg
Use of a weapon in which you're restricted or unskilled
will incur a modest penalty in the chance to hit a monster and also in
the amount of damage done when you do hit; at basic level, there is no
penalty or bonus; at skilled level, you receive a modest bonus in the
chance to hit and amount of damage done; at expert level, the bonus is
higher.  A successful hit has a chance to boost your training towards
the next skill level (unless you've already reached the limit for this
skill).  Once such training reaches the threshold for that next level,
you'll be told that you feel more confident in your skills.  At that
point you can use `{\tt \#enhance}' to increase one or more skills.  Such skills
are not increased automatically because there is a limit to your total
overall skills, so you need to actively choose which skills to enhance
and which to ignore.

%.hn 2
\subsection*{Armor (`{\tt [}')}

%.pg
Lots of unfriendly things lurk about; you need armor to protect
yourself from their blows.  Some types of armor offer better
protection than others.  Your armor class is a measure of this
protection.  Armor class (AC) is measured as in AD\&D, with 10 being
the equivalent of no armor, and lower numbers meaning better armor.
Each suit of armor which exists in AD\&D gives the same protection in
{\it NetHack}.  Here is an (incomplete) list of the armor classes provided by
various suits of armor:

\begin{center}
\begin{tabular}{lllll}
dragon scale mail      & 1 & \makebox[20mm]{}  & plate mail            & 3\\
crystal plate mail     & 3 &                   & bronze plate mail     & 4\\
splint mail            & 4 &                   & banded mail           & 4\\
dwarvish mithril-coat  & 4 &                   & elven mithril-coat    & 5\\
chain mail             & 5 &                   & orcish chain mail     & 6\\
scale mail             & 6 &                   & studded leather armor & 7\\
ring mail              & 7 &                   & orcish ring mail      & 8\\
leather armor          & 8 &                   & leather jacket        & 9\\
no armor               & 10
\end{tabular}
\end{center}

%.pg
\nd You can also wear other pieces of armor (ex.\ helmets, boots,
shields, cloaks)
to lower your armor class even further, but you can only wear one item
of each category (one suit of armor, one cloak, one helmet, one
shield, and so on) at a time.

%.pg
If a piece of armor is enchanted, its armor protection will be better
(or worse) than normal, and its ``plus'' (or minus) will subtract from
your armor class.  For example, a +1 chain mail would give you
better protection than normal chain mail, lowering your armor class one
unit further to 4.  When you put on a piece of armor, you immediately
find out the armor class and any ``plusses'' it provides.  Cursed
pieces of armor usually have negative enchantments (minuses) in
addition to being unremovable.

%.pg
Many types of armor are subject to some kind of damage like rust.  Such
damage can be repaired.  Some types of armor may inhibit spell casting.

%.pg
The commands to use armor are `{\tt W}' (wear) and `{\tt T}' (take off).
The `{\tt A}' command can also be used to take off armor as well as other
worn items.

%.hn 2
\subsection*{Food (`{\tt \%}')}

%.pg
Food is necessary to survive.  If you go too long without eating you
will faint, and eventually die of starvation.
Some types of food will spoil, and become unhealthy to eat,
if not protected.
Food stored in ice boxes or tins (``cans'')
will usually stay fresh, but ice boxes are heavy, and tins
take a while to open.

%.pg
When you kill monsters, they usually leave corpses which are also
``food.''  Many, but not all, of these are edible; some also give you
special powers when you eat them.  A good rule of thumb is ``you are
what you eat.''

%.pg
Some character roles and some monsters are vegetarian.  Vegetarian monsters
will typically never eat animal corpses, while vegetarian players can,
but with some rather unpleasant side-effects.

%.pg
You can name one food item after something you like to eat with the
{\it fruit\/} option.

%.pg
The command to eat food is `{\tt e}'.

%.hn 2
\subsection*{Scrolls (`{\tt ?}')}

%.pg
Scrolls are labeled with various titles, probably chosen by ancient wizards
for their amusement value (ex.\ ``READ ME,'' or ``THANX MAUD'' backwards).
Scrolls disappear after you read them (except for blank ones, without
magic spells on them).

%.pg
One of the most useful of these is the %
{\it scroll of identify}, which
can be used to determine what another object is, whether it is cursed or
blessed, and how many uses it has left.  Some objects of subtle
enchantment are difficult to identify without these.

%.pg
A mail daemon may run up and deliver mail to you as a %
{\it scroll of mail} (on versions compiled with this feature).
To use this feature on versions where {\it NetHack\/}
mail delivery is triggered by electronic mail appearing in your system mailbox,
you must let {\it NetHack\/} know where to look for new mail by setting the
``MAIL'' environment variable to the file name of your mailbox.
You may also want to set the ``MAILREADER'' environment variable to the
file name of your favorite reader, so {\it NetHack\/} can shell to it when you
read the scroll.
On versions of {\it NetHack\/} where mail is randomly
generated internal to the game, these environment variables are ignored.
You can disable the mail daemon by turning off the
{\it mail\/} option.

%.pg
The command to read a scroll is `{\tt r}'.

%.hn 2
\subsection*{Potions (`{\tt !}')}

%.pg
Potions are distinguished by the color of the liquid inside the flask.
They disappear after you quaff them.

%.pg
Clear potions are potions of water.  Sometimes these are
blessed or cursed, resulting in holy or unholy water.  Holy water is
the bane of the undead, so potions of holy water are good things to
throw (`{\tt t}') at them.  It is also sometimes very useful to dip
(``{\tt \#dip}'') an object into a potion.

%.pg
The command to drink a potion is `{\tt q}' (quaff).

%.hn 2
\subsection*{Wands (`{\tt /}')}

%.pg
Magic wands usually have multiple magical charges.  Some wands are
directional---you must give a direction in which to zap them.  You can also
zap them at yourself (just give a `{\tt .}' or `{\tt s}' for the direction).
Be warned, however, for this is often unwise.  Other wands are
nondirectional---they don't require a direction.  The number of charges in a
wand is random and decreases by one whenever you use it.

%.pg
When the number of charges left in a wand becomes zero, attempts to use the
wand will usually result in nothing happening.  Occasionally, however, it may
be possible to squeeze the last few mana points from an otherwise spent wand,
destroying it in the process.  A wand may be recharged by using suitable
magic, but doing so runs the risk of causing it to explode.  The chance
for such an explosion starts out very small and increases each time the
wand is recharged.

%.pg
In a truly desperate situation, when your back is up against the wall, you
might decide to go for broke and break your wand.  This is not for the faint
of heart.  Doing so will almost certainly cause a catastrophic release of
magical energies.

%.pg
When you have fully identified a particular wand, inventory display will
include additional information in parentheses: the number of times it has
been recharged followed by a colon and then by its current number of charges.
A current charge count of {\tt -1} is a special case indicating that the wand
has been cancelled.

%.pg
The command to use a wand is `{\tt z}' (zap).  To break one, use the `{\tt a}'
(apply) command.

%.hn 2
\subsection*{Rings (`{\tt =}')}

%.pg
Rings are very useful items, since they are relatively permanent
magic, unlike the usually fleeting effects of potions, scrolls, and
wands.

%.pg
Putting on a ring activates its magic.  You can wear only two
rings, one on each ring finger.

%.pg
Most rings also cause you to grow hungry more rapidly, the rate
varying with the type of ring.

%.pg
The commands to use rings are `{\tt P}' (put on) and `{\tt R}' (remove).

%.hn 2
\subsection*{Spellbooks (`{\tt +}')}

%.pg
Spellbooks are tomes of mighty magic.  When studied with the `{\tt r}' (read)
command, they transfer to the reader the knowledge of a spell (and
therefore eventually become unreadable) --- unless the attempt backfires.
Reading a cursed spellbook or one with mystic runes beyond
your ken can be harmful to your health!

%.pg
A spell (even when learned) can also backfire when you cast it.  If you
attempt to cast a spell well above your experience level, or if you have
little skill with the appropriate spell type, or cast it at
a time when your luck is particularly bad, you can end up wasting both the
energy and the time required in casting.

%.pg
Casting a spell calls forth magical energies and focuses them with
your naked mind.  Some of the magical energy released comes from within
you, and casting several spells in a row may tire you.
Casting of spells also requires practice.  With practice, your
skill in each category of spell casting will improve.  Over time, however,
your memory of each spell will dim, and you will need to relearn it.

%.pg
Some spells are
directional---you must give a direction in which to cast them.  You can also
cast them at yourself (just give a `{\tt .}' or `{\tt s}' for the direction).
Be warned, however, for this is often unwise.  Other spells are
nondirectional---they don't require a direction.

%.pg
Just as weapons are divided into groups in which a character can become
proficient (to varying degrees), spells are similarly grouped.
Successfully casting a spell exercises the skill group; sufficient skill
may increase the potency of the spell and reduce the risk of spell failure.
Skill slots are shared with weapons skills.  (See also the section on
``Weapon proficiency''.)

%.pg
Casting a spell also requires flexible movement, and wearing various types
of armor may interfere with that.

%.pg
The command to read a spellbook is the same as for scrolls, `{\tt r}'
(read).  The `{\tt +}' command lists your current spells, their levels,
categories, and chances for failure.
The `{\tt Z}' (cast) command casts a spell.

%.hn 2
\subsection*{Tools (`{\tt (}')}

%.pg
Tools are miscellaneous objects with various purposes.  Some tools
have a limited number of uses, akin to wand charges.  For example, lamps burn
out after a while.  Other tools are containers, which objects can
be placed into or taken out of.

%.pg
The command to use tools is `{\tt a}' (apply).

%.hn 3
\subsection*{Containers}

%.pg
You may encounter bags, boxes, and chests in your travels.  A tool of
this sort can be opened with the ``{\tt \#loot}'' extended command when
you are standing on top of it (that is, on the same floor spot),
or with the `{\tt a}' (apply) command when you are carrying it.  However,
chests are often locked, and are in any case unwieldy objects.
You must set one down before unlocking it by
using a key or lock-picking tool with the `{\tt a}' (apply) command,
by kicking it with the `{\tt \^{}D}' command,
or by using a weapon to force the lock with the ``{\tt \#force}''
extended command.

%.pg
Some chests are trapped, causing nasty things to happen when you
unlock or open them.  You can check for and try to deactivate traps
with the ``{\tt \#untrap}'' extended command.

%.hn 2
\subsection*{Amulets (`{\tt "}')}

%.pg
Amulets are very similar to rings, and often more powerful.  Like
rings, amulets have various magical properties, some beneficial,
some harmful, which are activated by putting them on.

%.pg
Only one amulet may be worn at a time, around your neck.

%.pg
The commands to use amulets are the same as for rings, `{\tt P}' (put on)
and `{\tt R}' (remove).

%.hn 2
\subsection*{Gems (`{\tt *}')}

%.pg
Some gems are valuable, and can be sold for a lot of gold.  They are also
a far more efficient way of carrying your riches.  Valuable gems increase
your score if you bring them with you when you exit.

%.pg
Other small rocks are also categorized as gems, but they are much less
valuable.  All rocks, however, can be used as projectile weapons (if you
have a sling).  In the most desperate of cases, you can still throw them
by hand.

%.hn 2
\subsection*{Large rocks (`{\tt `}')}
%.pg
Statues and boulders are not particularly useful, and are generally
heavy.  It is rumored that some statues are not what they seem.

%.pg
Very large humanoids (giants and their ilk) have been known to use boulders
as weapons.

%.hn 2
\subsection*{Gold (`{\tt \$}')}

%.pg
Gold adds to your score, and you can buy things in shops with it.
There are a number
of monsters in the dungeon that may be influenced by the amount of gold
you are carrying (shopkeepers aside).

%.hn 1
\section{Conduct}

%.pg
As if winning {\it NetHack\/} were not difficult enough, certain players
seek to challenge themselves by imposing restrictions on the
way they play the game.  The game automatically tracks some of
these challenges, which can be checked at any time with the {\tt \#conduct}
command or at the end of the game.  When you perform an action which
breaks a challenge, it will no longer be listed.  This gives
players extra ``bragging rights'' for winning the game with these
challenges.  Note that it is perfectly acceptable to win the game
without resorting to these restrictions and that it is unusual for
players to adhere to challenges the first time they win the game.

%.pg
Several of the challenges are related to eating behavior.  The most
difficult of these is the foodless challenge.  Although creatures
can survive long periods of time without food, there is a physiological
need for water; thus there is no restriction on drinking beverages,
even if they provide some minor food benefits.
Calling upon your god for help with starvation does
not violate any food challenges either.

%.pg
A strict vegan diet is one which avoids any food derived from animals.
The primary source of nutrition is fruits and vegetables.  The
corpses and tins of blobs (`b'), jellies (`j'), and fungi (`F') are
also considered to be vegetable matter.  Certain human
food is prepared without animal products; namely, lembas wafers, cram
rations, food rations (gunyoki), K-rations, and C-rations.
Metal or another normally indigestible material eaten while polymorphed
into a creature that can digest it is also considered vegan food.
Note however that eating such items still counts against foodless conduct.

%.pg
Vegetarians do not eat animals;
however, they are less selective about eating animal byproducts than vegans.
In addition to the vegan items listed above, they may eat any kind
of pudding (`P') other than the black puddings,
eggs and food made from eggs (fortune cookies and pancakes),
food made with milk (cream pies and candy bars), and lumps of
royal jelly.  Monks are expected to observe a vegetarian diet.

%.pg
Eating any kind of meat violates the vegetarian, vegan, and foodless
conducts.  This includes tripe rations, the corpses or tins of any
monsters not mentioned above, and the various other chunks of meat
found in the dungeon.  Swallowing and digesting a monster while polymorphed
is treated as if you ate the creature's corpse.
Eating leather, dragon hide, or bone items while
polymorphed into a creature that can digest it, or eating monster brains
while polymorphed into a mind flayer, is considered eating
an animal, although wax is only an animal byproduct.

%.pg
Regardless of conduct, there will be some items which are indigestible,
and others which are hazardous to eat.  Using a swallow-and-digest
attack against a monster is equivalent to eating the monster's corpse.
Please note that the term ``vegan'' is used here only in the context of
diet.  You are still free to choose not to use or wear items derived
from animals (e.g. leather, dragon hide, bone, horns, coral), but the
game will not keep track of this for you.  Also note that ``milky''
potions may be a translucent white, but they do not contain milk,
so they are compatible with a vegan diet.  Slime molds or
player-defined ``fruits'', although they could be anything
from ``cherries'' to ``pork chops'', are also assumed to be vegan.

%.pg
An atheist is one who rejects religion.  This means that you cannot
{\tt \#pray}, {\tt \#offer} sacrifices to any god,
{\tt \#turn} undead, or {\tt \#chat} with a priest.
Particularly selective readers may argue that playing Monk or Priest
characters should violate this conduct; that is a choice left to the
player.  Offering the Amulet of Yendor to your god is necessary to
win the game and is not counted against this conduct.  You are also
not penalized for being spoken to by an angry god, priest(ess), or
other religious figure; a true atheist would hear the words but
attach no special meaning to them.

%.pg
Most players fight with a wielded weapon (or tool intended to be
wielded as a weapon).  Another challenge is to win the game without
using such a wielded weapon.  You are still permitted to throw,
fire, and kick weapons; use a wand, spell, or other type of item;
or fight with your hands and feet.

%.pg
In {\it NetHack\/}, a pacifist refuses to cause the death of any other monster
(i.e. if you would get experience for the death).  This is a particularly
difficult challenge, although it is still possible to gain experience
by other means.

%.pg
An illiterate character cannot read or write.  This includes reading
a scroll, spellbook, fortune cookie message, or t-shirt; writing a
scroll; or making an engraving of anything other than a single ``x'' (the
traditional signature of an illiterate person).  Reading an engraving,
or any item that is absolutely necessary to win the game, is not counted
against this conduct.  The identity of scrolls and spellbooks (and
knowledge of spells) in your starting inventory is assumed to be
learned from your teachers prior to the start of the game and isn't
counted.

%.pg
There are several other challenges tracked by the game.  It is possible
to eliminate one or more species of monsters by genocide; playing without
this feature is considered a challenge.  When the game offers you an
opportunity to genocide monsters, you may respond with the monster type
``none'' if you want to decline.  You can change the form of an item into
another item of the same type (``polypiling'') or the form of your own
body into another creature (``polyself'') by wand, spell, or potion of
polymorph; avoiding these effects are each considered challenges.
Polymorphing monsters, including pets, does not break either of these
challenges.
Finally, you may sometimes receive wishes; a game without an attempt to
wish for any items is a challenge, as is a game without wishing for
an artifact (even if the artifact immediately disappears).  When the
game offers you an opportunity to make a wish for an item, you may
choose ``nothing'' if you want to decline.

%.hn 1
\section{Options}

%.pg
Due to variations in personal tastes and conceptions of how {\it NetHack\/}
should do things, there are options you can set to change how {\it NetHack\/}
behaves.

%.hn 2
\subsection*{Setting the options}

%.pg
Options may be set in a number of ways.  Within the game, the `{\tt O}'
command allows you to view all options and change most of them.
You can also set options automatically by placing them in the
``NETHACKOPTIONS'' environment variable or in a configuration file.
Some versions of {\it NetHack\/} also have front-end programs that allow
you to set options before starting the game.

%.hn 2
\subsection*{Using the NETHACKOPTIONS environment variable}

%.pg
The NETHACKOPTIONS variable is a comma-separated list of initial
values for the various options.  Some can only be turned on or off.
You turn one of these on by adding the name of the option to the list,
and turn it off by typing a `{\tt !}' or ``{\tt no}'' before the name.
Others take a
character string as a value.  You can set string options by typing
the option name, a colon or equals sign, and then the value of the string.
The value is terminated by the next comma or the end of string.

%.pg
For example, to set up an environment variable so that {\it autoquiver\/}
is on, {\it autopickup\/} is off, the {\it name\/} is set to ``Blue Meanie'',
and the {\it fruit\/} is set to ``papaya'', you would enter the command
%.sd
\begin{verbatim}
    setenv NETHACKOPTIONS "autoquiver,\!autopickup,name:Blue Meanie,fruit:papaya"
\end{verbatim}
%.ed

\nd in {\it csh}
(note the need to escape the ! since it's special to the shell), or
%.sd
\begin{verbatim}
    NETHACKOPTIONS="autoquiver,!autopickup,name:Blue Meanie,fruit:papaya"
    export NETHACKOPTIONS
\end{verbatim}
%.ed

\nd in {\it sh\/} or {\it ksh}.

%.hn 2
\subsection*{Using a configuration file}

%.pg
Any line in the configuration file starting with `{\tt \#}' is treated as a comment.
Any line in the configuration file starting with ``{\tt OPTIONS=}'' may be
filled out with options in the same syntax as in NETHACKOPTIONS.
Any line starting with ``{\tt DUNGEON=}'', ``{\tt EFFECTS=}'',
``{\tt MONSTERS=}'', ``{\tt OBJECTS=}'', ``{\tt TRAPS=}'', 
or ``{\tt BOULDER=}''
is taken as defining the corresponding {\it dungeon},
{\it effects}, {\it monsters}, {\it objects}, {\it traps\/} or
{\it boulder\/} option in a different syntax,
a sequence of decimal numbers giving the character position
in the current font to be used in displaying each entry.
A zero in any entry in such a sequence leaves the display of that
entry unchanged; this feature is not available using the option syntax.
Such a sequence can be continued to multiple lines by putting a
`{\tt \verb+\+}' at the end of each line to be continued.

%.pg
If your copy of the game included the compile time AUTOPICKUP\_EXCEPTIONS 
option, then any line starting with ``{\tt AUTOPICKUP\_EXCEPTION=}'' 
is taken as defining an exception to the ``{\tt pickup\_types}'' option.
There is a section of this Guidebook that discusses that.

%.pg
The default name of the configuration file varies on different
operating systems, but NETHACKOPTIONS can also be set to
the full name of a file you want to use (possibly preceded by an `{\tt @}').

%.hn 2
\subsection*{Customization options}

%.pg
Here are explanations of what the various options do.
Character strings that are too long may be truncated.
Some of the options listed may be inactive in your dungeon.

\blist{}
%.lp
\item[\ib{align}]
Your starting alignment ({\tt align:lawful}, {\tt align:neutral},
or {\tt align:chaotic}).  You may specify just the first letter.
The default is to randomly pick an appropriate alignment.
Cannot be set with the `{\tt O}' command.
%.lp
\item[\ib{autodig}]
Automatically dig if you are wielding a digging tool and moving into a place
that can be dug (default false).
%.lp
\item[\ib{autoopen}]
Walking into a door attempts to open it (default true).
%.lp
\item[\ib{autopickup}]
Automatically pick up things onto which you move (default on).
See ``{\it pickup\_types\/}'' to refine the behavior.
%.lp
\item[\ib{autoquiver}]
This option controls what happens when you attempt the `f' (fire)
command with an empty quiver.  When true, the computer will fill
your quiver with some suitable weapon.  Note that it will not take
into account the blessed/cursed status, enchantment, damage, or
quality of the weapon; you are free to manually fill your quiver with
the `Q' command instead.  If no weapon is found or the option is
false, the `t' (throw) command is executed instead.  (default false)
%.lp
\item[\ib{boulder}]
Set the character used to display boulders (default is rock class symbol).
%.lp
\item[\ib{catname}]
Name your starting cat (ex.\ ``{\tt catname:Morris}'').
Cannot be set with the `{\tt O}' command.
%.lp character
\item[\ib{character}]
Pick your type of character (ex.\ ``{\tt character:Monk}'');
synonym for ``{\it role\/}''.  See ``{\it name\/}'' for an alternate method
of specifying your role.  Normally only the first letter of
the value is examined; the string ``{\tt random}'' is an exception.
%.lp
\item[\ib{checkpoint}]
Save game state after each level change, for possible recovery after
program crash (default on).
%.lp
\item[\ib{checkspace}]
Check free disk space before writing files to disk (default on).
You may have to turn this off if you have more than 2 GB free space
on the partition used for your save and level files.
Only applies when MFLOPPY was defined during compilation.
%.lp
\item[\ib{cmdassist}]
Have the game provide some additional command assistance for new 
players if it detects some anticipated mistakes (default on).
%.lp
\item[\ib{confirm}]
Have user confirm attacks on pets, shopkeepers, and other
peaceable creatures (default on).
%.lp
\item[\ib{DECgraphics}]
Use a predefined selection of characters from the DEC VT-xxx/DEC
Rainbow/ANSI line-drawing character set to display the dungeon/effects/traps
instead of having to define a full graphics set yourself (default off).
This option also sets up proper handling of graphics
characters for such terminals, so you should specify it when appropriate
even if you override the selections with your own graphics strings.
%.lp
\item[\ib{disclose}]
Controls options for disclosing various information when the game ends (defaults
to all possibilities being disclosed).
The possibilities are:

%.sd
%.si
{\tt i} --- disclose your inventory.\\
{\tt a} --- disclose your attributes.\\
{\tt v} --- summarize monsters that have been vanquished.\\
{\tt g} --- list monster species that have been genocided.\\
{\tt c} --- display your conduct.
%.ei
%.ed

Each disclosure possibility can optionally be preceded by a prefix which
let you refine how it behaves. Here are the valid prefixes:

%.sd
%.si
{\tt y} --- prompt you and default to yes on the prompt.\\
{\tt n} --- prompt you and default to no on the prompt.\\
{\tt +} --- disclose it without prompting.\\
{\tt -} --- do not disclose it and do not prompt.
%.ei
%.ed

(ex.\ ``{\tt disclose:yi na +v -g -c}'')
The example sets {\it inventory\/} to {\it prompt\/} and default to {\it yes\/}, 
{\it attributes\/} to {\it prompt\/} and default to {\it no\/}, 
{\it vanquished\/} to {\it disclose without prompting\/}, 
{\it genocided\/} to {\it not disclose\/} and not to {\it prompt\/}, and 
{\it conduct\/} to {\it not disclose\/} and not to {\it prompt\/}.
Note that the vanquished monsters list includes all monsters killed by
traps and each other as well as by you. 
%.lp
\item[\ib{dogname}]
Name your starting dog (ex.\ ``{\tt dogname:Fang}'').
Cannot be set with the `{\tt O}' command.
%.lp
\item[\ib{dumpfile}]
The name of a file where the disclosure information is written when the
game ends. You may use the macro \%n that will be replaced with the name
of your player character. The game must have write permissions to the
directory where the file is written. Normally /tmp may be used for unix
systems.
%.lp
\item[\ib{dungeon}]
Set the graphics symbols for displaying the dungeon (default
``\verb& |--------||.-|++##& \verb&.##<><>_|\\#{}.}..## #}&'').
The {\it dungeon\/} option should be
followed by a string of 1--41
characters to be used instead of the default map-drawing characters.
The dungeon map will use the characters you specify instead of the
default symbols, and default symbols for any you do not specify.
Remember that you may need to escape some of these characters
on a command line if they are special to your shell.

Note that {\it NetHack\/} escape-processes this option string in conventional C
fashion.  This means that `\verb+\+' is a prefix to take the following
character literally.  Thus `\verb+\+' needs to be represented as `\verb+\\+'.
The special escape form
`\verb+\m+' switches on the meta bit in the following character, and the
`{\tt \^{}}' prefix causes the following character to be treated as a control
character.

The order of the symbols is:  solid rock, vertical wall, horizontal
wall, upper left corner, upper right corner, lower left corner, lower
right corner, cross wall, upward T wall, downward T wall, leftward T
wall, rightward T wall, no door, vertical open door, horizontal open
door, vertical closed door, horizontal closed door, iron bars, tree,
floor of a room, dark corridor, lit corridor, stairs up, stairs down,
ladder up, ladder down, altar, grave, throne, kitchen sink, fountain, pool or moat,
ice, lava, vertical lowered drawbridge, horizontal lowered drawbridge,
vertical raised drawbridge, horizontal raised drawbridge, air, cloud,
under water.

You might want to use `{\tt +}' for the corners and T walls for a more
aesthetic, boxier display.  Note that in the next release, new symbols
may be added, or the present ones rearranged.

Cannot be set with the `{\tt O}' command.
%.lp
\item[\ib{effects}]
Set the graphics symbols for displaying special effects (default
``\verb&|-\\/*!)(0#@*/-\\& \verb&||\\-//-\\| |\\-/&'').
The {\it effects\/} option should be
followed by a string of 1--29
characters to be used instead of the default special-effects characters.
This string is subjected to the same processing as the {\it dungeon\/} option.

The order of the symbols is:  vertical beam, horizontal beam, left slant,
right slant, digging beam, camera flash beam, left boomerang, right boomerang,
four glyphs giving the sequence for magic resistance displays,
the eight surrounding glyphs for swallowed display,
nine glyphs for explosions.
An explosion consists of three rows (top, middle, and bottom) of three
characters.  The explosion is centered in the center of this $3 \times 3$
array.

Note that in the next release, new symbols may be added,
or the present ones rearranged.

Cannot be set with the `{\tt O}' command.
%.lp
\item[\ib{extmenu}]
Changes the extended commands interface to pop-up a menu of available commands.
It is keystroke compatible with the traditional interface except that it does
not require that you hit Enter.  It is implemented only by the tty port 
(default off), when the game has been compiled to support tty graphics.
%.lp
\item[\ib{female}]
An obsolete synonym for ``{\tt gender:female}''.  Cannot be set with the
`{\tt O}' command.
%.lp
\item[\ib{fixinv}]
An object's inventory letter sticks to it when it's dropped (default on).
If this is off, dropping an object shifts all the remaining inventory letters.
%.lp
\item[\ib{fruit}]
Name a fruit after something you enjoy eating (ex.\ ``{\tt fruit:mango}'')
(default ``{\tt slime mold}''). Basically a nostalgic whimsy that
{\it NetHack\/} uses from time to time.  You should set this to something you
find more appetizing than slime mold.  Apples, oranges, pears, bananas, and
melons already exist in {\it NetHack}, so don't use those.
%.Ip
\item[\ib{gender}]
Your starting gender ({\tt gender:male} or {\tt gender:female}).
You may specify just the first letter.  Although you can
still denote your gender using the ``{\tt male}'' and ``{\tt female}''
options, the ``{\tt gender}'' option will take precedence.
The default is to randomly pick an appropriate gender.
Cannot be set with the `{\tt O}' command.
%.lp
\item[\ib{help}]
If more information is available for an object looked at
with the `{\tt /}' command, ask if you want to see it (default on).
Turning help off makes just looking at things faster, since you aren't
interrupted with the ``{\tt More info?}'' prompt, but it also means that you
might miss some interesting and/or important information.
%.lp
\item[\ib{horsename}]
Name your starting horse (ex.\ ``{\tt horsename:Trigger}'').
Cannot be set with the `{\tt O}' command.
%.lp
\item[\ib{IBMgraphics}]
Use a predefined selection of IBM extended ASCII characters to display the
dungeon/effects/traps instead of having to define a full graphics set
yourself (default off).
This option also sets up proper handling of graphics
characters for such terminals, so you should specify it when appropriate
even if you override the selections with your own graphics strings.
%.lp
\item[\ib{ignintr}]
Ignore interrupt signals, including breaks (default off).
%.lp
\item[\ib{legacy}]
Display an introductory message when starting the game (default on).
%.lp
\item[\ib{lit\_corridor}]
Show corridor squares seen by night vision or a light source held by your
character as lit (default off).
%.lp
\item[\ib{lootabc}]
Use the old `{\tt a}', `{\tt b}', and `{\tt c}' keyboard shortcuts when
looting, rather than the mnemonics `{\tt o}', `{\tt i}', and `{\tt b}' (default off).
%.lp
\item[\ib{mail}]
Enable mail delivery during the game (default on).
%.lp
\item[\ib{male}]
An obsolete synonym for ``{\tt gender:male}''.  Cannot be set with the
`{\tt O}' command.
%.lp
\item[\ib{menustyle}]
Controls the interface used when you need to choose various objects (in
response to the Drop command, for instance).  The value specified should
be the first letter of one of the following:  traditional, combination,
partial, or full.  Traditional was the only interface available for
earlier versions; it consists of a prompt for object class characters,
followed by an object-by-object prompt for all items matching the selected
object class(es).  Combination starts with a prompt for object class(es)
of interest, but then displays a menu of matching objects rather than
prompting one-by-one.  Partial skips the object class filtering and
immediately displays a menu of all objects.  Full displays a menu of
object classes rather than a character prompt, and then a menu of matching
objects for selection.
\item[\ib{menu\_deselect\_all}]
Menu character accelerator to deselect all items in a menu.
Implemented by the Amiga, Gem, X11 and tty ports.
Default `-'.
\item[\ib{menu\_deselect\_page}]
Menu character accelerator to deselect all items on this page of a menu.
Implemented by the Amiga, Gem and tty ports.
Default `\verb+\+'.
\item[\ib{menu\_first\_page}]
Menu character accelerator to jump to the first page in a menu.
Implemented by the Amiga, Gem and tty ports.
Default `\verb+^+'.
\item[\ib{menu\_headings}]
Controls how the headings in a menu are highlighted.
Values are ``{\tt bold}'', ``{\tt inverse}'', or ``{\tt underline}''.
Not all ports can actually display all three types.
\item[\ib{menu\_invert\_all}]
Menu character accelerator to invert all items in a menu.
Implemented by the Amiga, Gem, X11 and tty ports.
Default `@'.
\item[\ib{menu\_invert\_page}]
Menu character accelerator to invert all items on this page of a menu.
Implemented by the Amiga, Gem and tty ports.
Default `\verb+~+'.
\item[\ib{menu\_last\_page}]
Menu character accelerator to jump to the last page in a menu.
Implemented by the Amiga, Gem and tty ports.
Default `\verb+|+'.
\item[\ib{menu\_next\_page}]
Menu character accelerator to goto the next menu page.
Implemented by the Amiga, Gem and tty ports.
Default `\verb+>+'.
\item[\ib{menu\_previous\_page}]
Menu character accelerator to goto the previous menu page.
Implemented by the Amiga, Gem and tty ports.
Default `\verb+<+'.
\item[\ib{menu\_search}]
Menu character accelerator to search for a menu item.
Implemented by the Amiga, Gem and X11 ports.
Default `:'.
\item[\ib{menu\_select\_all}]
Menu character accelerator to select all items in a menu.
Implemented by the Amiga, Gem, X11 and tty ports.
Default `.'.
\item[\ib{menu\_select\_page}]
Menu character accelerator to select all items on this page of a menu.
Implemented by the Amiga, Gem and tty ports.
Default `,'.
%.lp
\item[\ib{monsters}]
Set the characters used to display monster classes (default
``\verb+abcdefghijklmnopqrstuv+
\verb+wxyzABCDEFGHIJKLMNOPQRSTUVWXYZ@ '&;:~]+'').
This string is subjected to the same processing as the {\it dungeon\/} option.
The order of the symbols is
ant or other insect, blob, cockatrice,
dog or other canine, eye or sphere, feline,
gremlin, humanoid, imp or minor demon,
jelly, kobold, leprechaun,
mimic, nymph, orc,
piercer, quadruped, rodent,
arachnid or centipede, trapper or lurker above, horse or unicorn,
vortex, worm, xan or other mythical/fantastic insect,
light, zruty,
angelic being, bat or bird, centaur,
dragon, elemental, fungus or mold,
gnome, giant humanoid, invisible monster,
jabberwock, Keystone Kop, lich,
mummy, naga, ogre,
pudding or ooze, quantum mechanic, rust monster,
snake, troll, umber hulk,
vampire, wraith, xorn,
apelike creature, zombie,
human, ghost, golem,
demon, sea monster, lizard,
long worm tail, and mimic.
Cannot be set with the `{\tt O}' command.
%.lp
\item[\ib{msghistory}]
The number of top line messages to save (and recall with `{\tt \^{}P}')
(default 20). Cannot be set with the `{\tt O}' command.
%.lp
\item[\ib{msg\_window}]
Allows you to change the way recalled messages are displayed.
(It is currently implemented for tty only.) The possible values are:

%.sd
%.si
{\tt s} --- single message (default, this was the behavior before 3.4.0).\\
{\tt c} --- combination, two messages as {\it single\/}, then as {\it full\/}.\\
{\tt f} --- full window, oldest message first.\\
{\tt r} --- full window, newest message first.
%.ei
%.ed

For backward compatibility, no value needs to be specified (which
defaults to {\it full\/}), or it can be negated (which defaults
to {\it single\/}). 
%.lp
\item[\ib{name}]
Set your character's name (defaults to your user name).  You can also
set your character's role by appending a dash and one or more letters of
the role (that is, by suffixing one of
``{\tt -A -B -C -H -K -M -P -Ra -Ro -S -T -V -W}'').
If ``{\tt -@}'' is used for the role, then a random one will be
automatically chosen.
Cannot be set with the `{\tt O}' command.
%.lp
\item[\ib{newcolors}]
Set the color palette for tty ports to more beautiful values
than the original {\it NetHack\/} palette. If black is not shown as dark gray,
disable this option (default on).
Cannot be set with the `{\tt O}' command.
%.lp
\item[\ib{news}]
Read the {\it NetHack\/} news file, if present (default on).
Since the news is shown at the beginning of the game, there's no point
in setting this with the `{\tt O}' command.
%.lp
\item[\ib{null}]
Send padding nulls to the terminal (default off).
%.lp
\item[\ib{number\_pad}]
Use the number keys to move instead of {\tt [yuhjklbn]} (default 0 or off).
(number\_pad:2 invokes the old DOS behavior where `{\tt 5}' means `{\tt g}', 
meta-`{\tt 5}' means `{\tt G}',  and meta-`{\tt 0}' means `{\tt I}'.)
%.lp
\item[\ib{objects}]
Set the characters used to display object classes (default
``\verb&])[="(%!?+/$*`0_.&'').
This string is subjected to the same processing as the {\it dungeon\/} option.
The order of the symbols is
illegal-object (should never be seen), weapon, armor, ring, amulet, tool,
food, potion, scroll, spellbook, wand, gold, gem or rock, boulder or statue,
iron ball, chain, and venom.
Cannot be set with the `{\tt O}' command.
%.lp
\item[\ib{packorder}]
Specify the order to list object types in (default
``\verb&")[%?+!=/(*`0_&''). The value of this option should be a string
containing the symbols for the various object types.  Any omitted types
are filled in at the end from the previous order.
%.lp
\item[\ib{paranoid\_hit}]
If true, asks you to type the word ``yes'' when hitting any peaceful
monster, not just the letter ``y''.
%.lp
\item[\ib{paranoid\_quit}]
If true, asks you to type the word ``yes'' when quitting or entering
Explore mode, not just the letter ``y''.
%.lp
\item[\ib{paranoid\_remove}]
If true, always show menu with the R and T commands even when there is
only one item to remove or take off.
%.lp
\item[\ib{perm\_invent}]
If true, always display your current inventory in a window.  This only
makes sense for windowing system interfaces that implement this feature.
%.lp
\item[\ib{pettype}]
Specify the type of your initial pet, if you are playing a character class
that uses multiple types of pets; or choose to have no initial pet at all.
Possible values are ``{\tt cat}'', ``{\tt dog}'' and ``{\tt none}''.
Cannot be set with the `{\tt O}' command.
%.Ip
\item[\ib{pickup\_burden}]
When you pick up an item that would exceed this encumbrance
level (Unburdened, Burdened, streSsed, straiNed, overTaxed,
or overLoaded), you will be asked if you want to continue.
(Default `S').
%.lp
\item[\ib{pickup\_dropped}]
If this boolean option is false and {\it autopickup\/} +is on,
don't pick up things that you have droppped deliberately, even if they
are in {\it pickup\_types\/}.
%.lp
\item[\ib{pickup\_thrown}]
If this boolean option is true and {\it autopickup\/} is on, try to pick up
things that you threw, even if they aren't in {\it pickup\_types\/}.
Default is on.
%.lp
\item[\ib{pickup\_types}]
Specify the object types to be picked up when ``{\it autopickup\/}'' 
is on.  Default is all types.  If your copy of the game has the
experimental compile time option AUTOPICKUP\_EXCEPTIONS included,
you may be able to use ``{\it autopickup\_exception\/}'' configuration
file lines to further refine ``{\it autopickup\/}'' behavior.
%.lp
\item[\ib{pilesize}]
Allows setting the number of items that trigger the "Things that are here".
The minimum value allowed is 1, which will tell you about one object
and will never give you the "Things that are here" list. 
%.lp
\item[\ib{prayconfirm}]
Prompt for confirmation before praying (default on).
%.lp
\item[\ib{pushweapon}]
Using the `w' (wield) command when already wielding
something pushes the old item into your alternate weapon slot (default off).
%.Ip
\item[\ib{qwertz\_layout}]
If {\tt QWERTZ} was set in {\it config.h} at compile-time this option can be
used to swap the y and z keys. This is useful for some keyboard layouts.
%.Ip
\item[\ib{race}]
Selects your race (for example, ``{\tt race:human}'').  Default is random.
Cannot be set with the `{\tt O}' command.
%.lp
\item[\ib{rest\_on\_space}]
Make the space bar a synonym for the `{\tt .}' (rest) command (default off).
%.lp
\item[\ib{role}]
Pick your type of character (ex.\ ``{\tt role:Samurai}'');
synonym for ``{\it character\/}''.  See ``{\it name\/}'' for an alternate method
of specifying your role.  Normally only the first letter of the
value is examined; `r' is an exception with ``{\tt Rogue}'', {\tt Ranger}'',
and ``{\tt random}'' values.
%.lp
\item[\ib{runmode}]
Controls the amount of screen updating for the map window when engaged
in multi-turn movement (running via {\tt shift}+direction
or {\tt control}+direction
and so forth, or via the travel command or mouse click).
The possible values are:

%.sd
%.si
{\tt teleport} --- update the map after movement has finished;\\
{\tt run} --- update the map after every seven or so steps;\\
{\tt walk} --- update the map after each step;\\
{\tt crawl} --- like {\it walk\/}, but pause briefly after each step.
%.ei
%.ed

This option only affects the game's screen display, not the actual
results of moving.  The default is {\it run\/}; versions prior to 3.4.1 
used {\it teleport\/} only.  Whether or not the effect is noticeable will
depend upon the window port used or on the type of terminal.
%.lp
\item[\ib{safe\_pet}]
Prevent you from (knowingly) attacking your pets (default on).
%.lp
\item[\ib{scores}]
Control what parts of the score list you are shown at the end (ex.\
``{\tt scores:5top scores/4around my score/own scores}'').  Only the first
letter of each category (`{\tt t}', `{\tt a}' or `{\tt o}') is necessary.
%.lp
\item[\ib{showborn}]
When the game ends, show the number of each monster created
in the ``Vanquished creatures'' list, if it differs from the
number of those monsters killed.
%.lp
\item[\ib{showexp}]
Show your accumulated experience points on bottom line (default off).
%.lp
\item[\ib{showrace}]
Display yourself as the glyph for your race, rather than the glyph
for your role (default off).  Note that this setting affects only
the appearance of the display, not the way the game treats you.
%.lp
\item[\ib{showscore}]
Show your approximate accumulated score on bottom line (default off).
%.lp
\item[\ib{silent}]
Suppress terminal beeps (default on).
%.lp
\item[\ib{sortpack}]
Sort the pack contents by type when displaying inventory (default on).
%.lp
\item[\ib{sound}]
Enable messages about what your character hears (default on).
Note that this has nothing to do with your computer's audio capabilities.
This option is only partly under player control.  The game toggles it
off and on during and after sleep, for example.
%.lp
\item[\ib{standout}]
Boldface monsters and ``{\tt --More--}'' (default off).
%.lp
\item[\ib{sparkle}]
Display a sparkly effect when a monster (including yourself) is hit by an
attack to which it is resistant (default on).
%.lp
\item[\ib{suppress\_alert}]
This option may be set to a NetHack version level to suppress
alert notification messages about feature changes for that 
and prior versions (ex.\ ``{\tt suppress\_alert:3.3.1}'')
%.lp
\item[\ib{time}]
Show the elapsed game time in turns on bottom line (default off).
%.lp
\item[\ib{timed\_delay}]
When pausing momentarily for display effect, such as with explosions and
moving objects, use a timer rather than sending extra characters to the
screen.  (Applies to ``tty'' interface only; ``X11'' interface always
uses a timer based delay.  The default is on if configured into the
program.)
%.lp
\item[\ib{tombstone}]
Draw a tombstone graphic upon your death (default on).
%.lp
\item[\ib{toptenwin}]
Put the ending display in a NetHack window instead of on stdout (default off).
Setting this option makes the score list visible when a windowing version
of NetHack is started without a parent window, but it no longer leaves
the score list around after game end on a terminal or emulating window.
%.lp
\item[\ib{traps}]
Set the graphics symbols for displaying traps (default
``\verb&^^^^^^^^^^^^^^^^^"^^^^&'').
The {\it traps\/} option should be followed by a string of 1--22
characters to be used instead of the default traps characters.
This string is subjected to the same processing as the {\it dungeon\/} option.

The order of the symbols is:
arrow trap, dart trap, falling rock trap, squeaky board, bear trap,
land mine, rolling boulder trap, sleeping gas trap, rust trap, fire trap,
pit, spiked pit, hole, trap door, teleportation trap, level teleporter,
magic portal, web, statue trap, magic trap, anti-magic field, polymorph trap.

Cannot be set with the `{\tt O}' command.
%.lp
\item[\ib{travel}]
Allow the travel command (default on).  Turning this option off will
prevent the game from attempting unintended moves if you make inadvertent
mouse clicks on the map window.
%.lp
\item[\ib{verbose}]
Provide more commentary during the game (default on).
%.lp
\item[\ib{windowtype}]
Select which windowing system to use, such as ``{\tt tty}'' or ``{\tt X11}''
(default depends on version).
Cannot be set with the `{\tt O}' command.
%.lp
\item[\ib{win_edge}]
Align menus and text windows in tty left instead of right. (default off)
\elist

%.hn 2
\subsection*{Window Port Customization options}

%.pg
Here are explanations of the various options that are
used to customize and change the characteristics of the
windowtype that you have chosen.
Character strings that are too long may be truncated.
Not all window ports will adjust for all settings listed
here.  You can safely add any of these options to your 
config file, and if the window port is capable of adjusting 
to suit your preferences, it will attempt to do so. If it
can't it will silently ignore it.  You can find out if an 
option is supported by the window port that you are currently
using by checking to see if it shows up in the Options list.
Some options are dynamic and can be specified during the game
with the `{\tt O}' command.

\blist{}
%.lp
\item[\ib{align\_message}]
 Where to align or place the message window (top, bottom, left, or right)
%.lp
\item[\ib{align\_status}]
 Where to align or place the status window (top, bottom, left, or right).
%.lp
\item[\ib{ascii\_map}]
NetHack should display an ascii map if it can.
%.lp
\item[\ib{color}]
NetHack should display color if it can for different monsters, 
objects, and dungeon features
%.lp
\item[\ib{eight\_bit\_tty}]
Pass eight-bit character values (for example, specified with the {\it
traps \/} option) straight through to your terminal (default off).
%.lp
\item[\ib{font\_map}]
NetHack should use a font by the chosen name for the map window.
%.lp
\item[\ib{font\_menu}]
NetHack should use a font by the chosen name for menu windows.
%.lp
\item[\ib{font\_message}]
NetHack should use a font by the chosen name for the message window.
%.lp
\item[\ib{font\_status}]
NetHack should use a font by the chosen name for the status window.
%.lp
\item[\ib{font\_text}]
NetHack should use a font by the chosen name for text windows.
%.lp
\item[\ib{font\_size\_map}]
NetHack should use this size font for the map window.
%.lp
\item[\ib{font\_size\_menu}]
NetHack should use this size font for menu windows.
%.lp
\item[\ib{font\_size\_message}]
NetHack should use this size font for the message window.
%.lp
\item[\ib{font\_size\_status}]
NetHack should use this size font for the status window.
%.lp
\item[\ib{font\_size\_text}]
NetHack should use this size font for text windows.
%.lp
\item[\ib{fullscreen}]
NetHack should try and display on the entire screen rather than in a window.
%.lp
\item[\ib{hilite\_pet}]
Visually distinguish pets from similar animals (default off).
The behavior of this option depends on the type of windowing you use.
In text windowing, text highlighting or inverse video is often used;
with tiles, generally displays a heart symbol near pets.
%.lp
\item[\ib{large\_font}]
NetHack should use a large font.
%.lp
\item[\ib{map\_mode}]
NetHack should display the map in the manner specified.
%.lp
\item[\ib{mouse\_support}]
Allow use of the mouse for input and travel.
%.lp
\item[\ib{player\_selection}]
NetHack should pop up dialog boxes or use prompts for character selection.
%.lp
\item[\ib{popup\_dialog}]
NetHack should pop up dialog boxes for input.
%.lp
\item[\ib{preload\_tiles}]
NetHack should preload tiles into memory.
For example, in the protected mode MSDOS version, control whether tiles
get pre-loaded into RAM at the start of the game.  Doing so
enhances performance of the tile graphics, but uses more memory. (default on).
Cannot be set with the `{\tt O}' command.
%.lp
\item[\ib{scroll\_amount}]
NetHack should scroll the display by this number of cells
when the hero reaches the scroll\_margin.
%.lp
\item[\ib{scroll\_margin}]
NetHack should scroll the display when the hero or cursor
is this number of cells away from the edge of the window.
%.lp
\item[\ib{softkeyboard}]
Display an onscreen keyboard.  Handhelds are most likely to support this option.
%.lp
\item[\ib{splash\_screen}]
NetHack should display an opening splash screen when it starts up (default yes).
%.lp
\item[\ib{tiled\_map}]
NetHack should display a tiled map if it can.
%.lp
\item[\ib{tile\_file}]
Specify the name of an alternative tile file to override the default.
%.lp
\item[\ib{tile\_height}]
Specify the preferred height of each tile in a tile capable port.
%.lp
\item[\ib{tile\_width}]
Specify the preferred width of each tile in a tile capable port
%.lp
\item[\ib{use\_inverse}]
NetHack should display inverse when the game specifies it.
%.lp
\item[\ib{vary\_msgcount}]
NetHack should display this number of messages at a time in the message window.
%.lp
\item[\ib{windowcolors}]
NetHack should display windows with the specified foreground/background 
colors if it can.
%.lp
\item[\ib{wraptext}]
NetHack port should wrap long lines of text if they don't fit in 
the visible area of the window.
\elist

%.hn 2
\subsection*{Platform-specific Customization options}

%.pg
Here are explanations of options that are used by specific platforms 
or ports to customize and change the port behavior.

\blist{}
%.lp
\item[\ib{altkeyhandler}]
Select an alternate keystroke handler dll to load ({\it Win32 tty\/ NetHack\/} only).
The name of the handler is specified without the .dll extension and without any
path information.
Cannot be set with the `{\tt O}' command.
%.lp 
\item[\ib{altmeta}]
(default on, {\it Amiga NetHack \/} only).
%.lp
\item[\ib{BIOS}]
Use BIOS calls to update the screen display quickly and to read the keyboard
(allowing the use of arrow keys to move) on machines with an IBM PC
compatible BIOS ROM (default off, {\it OS/2, PC\/ {\rm and} ST NetHack\/} only).
%.lp 
\item[\ib{flush}]
(default off, {\it Amiga NetHack \/} only).
%.lp 
\item[\ib{Macgraphics}]
(default on, {\it Mac NetHack \/} only).
%.lp 
\item[\ib{page\_wait}]
(default off, {\it Mac NetHack \/} only).
%.lp
\item[\ib{rawio}]
Force raw (non-cbreak) mode for faster output and more
bulletproof input (MS-DOS sometimes treats `{\tt \^{}P}' as a printer toggle
without it) (default off, {\it OS/2, PC\/ {\rm and} ST NetHack\/} only).  
Note:  DEC Rainbows hang if this is turned on.
Cannot be set with the `{\tt O}' command.
%.lp
\item[\ib{soundcard}]
(default off, {\it PC NetHack \/} only).
Cannot be set with the `{\tt O}' command.
%.lp
\item[\ib{subkeyvalue}]
({\it Win32 tty NetHack \/} only).
May be used to alter the value of keystrokes that the operating system
returns to NetHack to help compensate for international keyboard issues.
OPTIONS=subkeyvalue:171/92
will return 92 to NetHack, if 171 was originally going to be returned.
You can use multiple subkeyvalue statements in the config file if needed.
Cannot be set with the `{\tt O}' command.
%.lp
\item[\ib{video}]
Set the video mode used ({\it PC\/ NetHack\/} only).
Values are {\it autodetect\/}, {\it default\/}, or {\it vga\/}. 
Setting {\it vga\/} (or {\it autodetect\/} with vga hardware present) will cause
the game to display tiles. 
Cannot be set with the `{\tt O}' command.
%.lp
\item[\ib{videocolors}]
\begin{sloppypar}
Set the color palette for PC systems using NO\_TERMS
(default 4-2-6-1-5-3-15-12-10-14-9-13-11, {\it PC\/ NetHack\/} only).
The order of colors is red, green, brown, blue, magenta, cyan,
bright.white, bright.red, bright.green, yellow, bright.blue,
bright.magenta, and bright.cyan.
Other systems compiled with TTY\_GRAPHICS and VIDEOSHADES allow defining 
three more colors. The order of colors is red, green, brown, blue, 
magenta, cyan, gray, black, bright.red, bright.green, yellow, 
bright.blue, bright.magenta, bright.cyan and white.
Cannot be set with the `{\tt O}' command.
\end{sloppypar}
%.lp
\item[\ib{videoshades}]
Set the intensity level of the three gray scales available
(default dark normal light, {\it PC\/ NetHack\/} only).
If the game display is difficult to read, try adjusting these scales;
if this does not correct the problem, try {\tt !color}.
Cannot be set with the `{\tt O}' command.
\elist

%.lp
%.hn 2
\subsection*{Configuring autopickup exceptions}

%.pg
There is an experimental compile time option called AUTOPICKUP_EXCEPTIONS.  
If your copy of the game was built with that option defined, you can 
further refine the behavior of the ``{\tt autopickup}'' option beyond 
what is available through the ``{\tt pickup\_types}'' option.

%.pg
By placing ``{\tt autopickup\_exception}'' lines in your configuration
file, you can define patterns to be checked when the game is about to
autopickup something.

\blist{}
%.lp
\item[\ib{autopickup\_exception}]
Sets an exception to the `{\it pickup\_types}' option.
The {\it autopickup\_exception\/} option should be followed by a string of 1--80
characters to be used as a pattern to match against the singular form
of the description of an object at your location.

%.pg
You may use the following special characters in a pattern:

\begin{verbatim}
    *--- matches 0 or more characters.
    ?--- matches any single character.
\end{verbatim}

In addition, some characters are treated specially if they occur as the first 
character in the specified string pattern, specifically:

%.sd
%.si
{\tt <} --- always pickup an object that matches the pattern that follows.\\
{\tt >} --- never pickup an object that matches the pattern that follows.
%.ei
%.ed

Can be set with the `{\tt O}' command, but the setting is not preserved
across saves and restores.
\elist

%.pg
Here's a couple of examples of autopickup\_exceptions:
\begin{verbatim}
    autopickup_exception="<*arrow"
    autopickup_exception=">*corpse"
    autopickup_exception=">* cursed*"
\end{verbatim}

The first example above will result in autopickup of any type of arrow.
The second example results in the exclusion of any corpse from autopickup.
The last example results in the exclusion of items known to be cursed from autopickup.
A `never pickup' rule takes precedence over an `always pickup' rule if both match.

%.lp
%.hn 2
\subsection*{Configuring User Sounds}

%.pg
Some platforms allow you to define sound files to be played when a message 
that matches a user-defined pattern is delivered to the message window.
At this time the Qt port and the win32tty and win32gui ports support the
use of user sounds.

%.pg
The following config file entries are relevant to mapping user sounds
to messages:

\blist{}
%.lp
\item[\ib{SOUNDDIR}]
The directory that houses the sound files to be played.
%.lp
\item[\ib{SOUND}]
An entry that maps a sound file to a user-specified message pattern.
Each SOUND entry is broken down into the following parts:

%.sd
%.si
{\tt MESG      } --- message window mapping (the only one supported in 3.4).\\
{\tt pattern   } --- the pattern to match.\\
{\tt sound file} --- the sound file to play.\\
{\tt volume    } --- the volume to be set while playing the sound file.
%.ei
%.ed
\elist

%.pg
The exact format for the pattern depends on whether the platform is
built to use {\it regular expressions \/} or NetHack's own internal pattern 
matching facility. The {\it regular expressions \/} matching can be much more 
sophisticated than the internal NetHack pattern matching, but requires 
3rd party libraries on some platforms.  There are plenty of references 
available elsewhere for explaining {\it regular expressions \/}. You can verify 
which pattern matching is used by your port with the 
\#version command.  

%.pg
NetHack's internal pattern matching routine uses the following
special characters in its pattern matching:

\begin{verbatim}
    *--- matches 0 or more characters.
    ?--- matches any single character.
\end{verbatim}

%.pg
Here's an example of a sound mapping using NetHack's internal
pattern matching facility:
\begin{verbatim}
    SOUND=MESG "*chime of a cash register*" "gong.wav" 50
\end{verbatim}
specifies that any message with "chime of a cash register" contained
in it will trigger the playing of "gong.wav".  You can have multiple
SOUND entries in your config file.

%.lp
%.hn 2
\subsection*{Configuring NetHack for Play by the Blind}

%.pg
NetHack can be set up to use only standard ASCII characters for making
maps of the dungeons. This makes the MS-DOS versions of NetHack completely
accessible to the blind who use speech and/or Braille access technologies.
Players will require a good working knowledge of their screen-reader's
review features, and will have to know how to navigate horizontally and
vertically character by character. They will also find the search
capabilities of their screen-readers to be quite valuable. Be certain to
examine this Guidebook before playing so you have an idea what the screen
layout is like. You'll also need to be able to locate the PC cursor. It is
always where your character is located. Merely searching for an @-sign will
not always find your character since there are other humanoids represented
by the same sign. Your screen-reader should also have a function which
gives you the row and column of your review cursor and the PC cursor.
These co-ordinates are often useful in giving players a better sense of the
overall location of items on the screen.
%.pg
While it is not difficult for experienced users to edit the {\it defaults.nh\/}
file to accomplish this, novices may find this task somewhat daunting.
Included in all official distributions of NetHack is a file called
{\it NHAccess.nh\/}.  Replacing {\it defaults.nh\/} with this file will cause
the game to run in a manner accessible to the blind. After you have gained
some experience with the game and with editing files, you may want to alter
settings to better suit your preferences. Instructions on how to do this
are included in the {\it NHAccess.nh\/} file itself. The most crucial
settings to make the game accessible are:
%.pg
\blist{}
%.lp
\item[\ib{IBMgraphics}]
Disable IBMgraphics by commenting out this option.
%.lp
\item[\ib{menustyle:traditional}]
This will assist in the interface to speech synthesizers.
%.lp
\item[\ib{number\_pad}]
A lot of speech access programs use the number-pad to review the screen.
If this is the case, disable the number\_pad option and use the traditional
Rogue-like commands.
%.lp
\item[\ib{Character graphics}]
Comment out all character graphics sets found near the bottom of the
{\it defaults.nh\/} file.  Most of these replace {\it NetHack\/}'s
default representation of the dungeon using standard ASCII characters
with fancier characters from extended character sets, and these fancier
characters can annoy screen-readers.
\elist

%.hn 1
\section{Scoring}

%.pg
{\it NetHack\/} maintains a list of the top scores or scorers on your machine,
depending on how it is set up.  In the latter case, each account on
the machine can post only one non-winning score on this list.  If
you score higher than someone else on this list, or better your
previous score, you will be inserted in the proper place under your
current name.  How many scores are kept can also be set up when
{\it NetHack\/} is compiled.

%.pg
Your score is chiefly based upon how much experience you gained, how
much loot you accumulated, how deep you explored, and how the game
ended.  If you quit the game, you escape with all of your gold intact.
If, however, you get killed in the Mazes of Menace, the guild will
only hear about 90\,\% of your gold when your corpse is discovered
(adventurers have been known to collect finder's fees).  So, consider
whether you want to take one last hit at that monster and possibly
live, or quit and stop with whatever you have.  If you quit, you keep
all your gold, but if you swing and live, you might find more.

%.pg
If you just want to see what the current top players/games list is, you
can type
\begin{verbatim}
    nethack -s all
\end{verbatim}
on most versions.

%.hn 1
\section{Explore mode}

%.pg
{\it NetHack\/} is an intricate and difficult game.  Novices might falter
in fear, aware of their ignorance of the means to survive.  Well, fear
not.  Your dungeon may come equipped with an ``explore'' or ``discovery''
mode that enables you to keep old save files and cheat death, at the
paltry cost of not getting on the high score list.

%.pg
There are two ways of enabling explore mode.  One is to start the game
with the {\tt -X}
switch.  The other is to issue the `{\tt X}' command while already playing
the game.  The other benefits of explore mode are left for the trepid
reader to discover.

%.hn
\section{Credits}
%.pg
The original %
{\it hack\/} game was modeled on the Berkeley
%.ux
UNIX
{\it rogue\/} game.  Large portions of this paper were shamelessly
cribbed from %
{\it A Guide to the Dungeons of Doom}, by Michael C. Toy
and Kenneth C. R. C. Arnold.  Small portions were adapted from
{\it Further Exploration of the Dungeons of Doom}, by Ken Arromdee.

%.pg
{\it NetHack\/} is the product of literally dozens of people's work.
Main events in the course of the game development are described below:

%.pg
\bigskip
\nd {\it Jay Fenlason\/} wrote the original {\it Hack\/} with help from {\it
Kenny Woodland}, {\it Mike Thome}, and {\it Jon Payne}.

%.pg
\medskip
\nd {\it Andries Brouwer\/} did a major re-write, transforming {\it Hack\/}
into a very different game, and published (at least) three versions (1.0.1,
1.0.2, and 1.0.3) for UNIX machines to the Usenet.

%.pg
\medskip
\nd {\it Don G. Kneller\/} ported {\it Hack\/} 1.0.3 to Microsoft C and MS-DOS,
producing {\it PC Hack\/} 1.01e, added support for DEC Rainbow graphics in
version 1.03g, and went on to produce at least four more versions (3.0, 3.2,
3.51, and 3.6).

%.pg
\medskip
\nd {\it R. Black\/} ported {\it PC Hack\/} 3.51 to Lattice C and the Atari
520/1040ST, producing {\it ST Hack\/} 1.03.

%.pg
\medskip
\nd {\it Mike Stephenson\/} merged these various versions back together,
incorporating many of the added features, and produced {\it NetHack\/} version
1.4.  He then coordinated a cast of thousands in enhancing and debugging
{\it NetHack\/} 1.4 and released {\it NetHack\/} versions 2.2 and 2.3.

%.pg
\medskip
\nd Later, Mike coordinated a major rewrite of the game, heading a team which
included {\it Ken Arromdee}, {\it Jean-Christophe Collet}, {\it Steve Creps},
{\it Eric Hendrickson}, {\it Izchak Miller}, {\it Eric S. Raymond}, {\it John
Rupley}, {\it Mike Threepoint}, and {\it Janet Walz}, to produce {\it
NetHack\/} 3.0c.

%.pg
\medskip
\nd {\it NetHack\/} 3.0 was ported to the Atari by {\it Eric R. Smith}, to OS/2 by
{\it Timo Hakulinen}, and to VMS by {\it David Gentzel}.  The three of them
and {\it Kevin Darcy\/} later joined the main development team to produce
subsequent revisions of 3.0.

%.pg
\medskip
\nd {\it Olaf Seibert\/} ported {\it NetHack\/} 2.3 and 3.0 to the Amiga.  {\it
Norm Meluch}, {\it Stephen Spackman\/} and {\it Pierre Martineau\/} designed
overlay code for {\it PC NetHack\/} 3.0.  {\it Johnny Lee\/} ported {\it
NetHack\/} 3.0 to the Macintosh.  Along with various other Dungeoneers, they
continued to enhance the PC, Macintosh, and Amiga ports through the later
revisions of 3.0.

%.pg
\medskip
\nd Headed by {\it Mike Stephenson\/} and coordinated by {\it Izchak Miller\/} and
{\it Janet Walz}, the development team which now included {\it Ken Arromdee},
{\it David Cohrs}, {\it Jean-Christophe Collet}, {\it Kevin Darcy},
{\it Matt Day}, {\it Timo Hakulinen}, {\it Steve Linhart}, {\it Dean Luick},
{\it Pat Rankin}, {\it Eric Raymond}, and {\it Eric Smith\/} undertook a radical
revision of 3.0.  They re-structured the game's design, and re-wrote major
parts of the code.  They added multiple dungeons, a new display, special
individual character quests, a new endgame and many other new features, and
produced {\it NetHack\/} 3.1.

%.pg
\medskip
\nd {\it Ken Lorber}, {\it Gregg Wonderly\/} and {\it Greg Olson}, with help
from {\it Richard Addison}, {\it Mike Passaretti}, and {\it Olaf Seibert},
developed {\it NetHack\/} 3.1 for the Amiga.

%.pg
\medskip
\nd {\it Norm Meluch\/} and {\it Kevin Smolkowski}, with help from
{\it Carl Schelin}, {\it Stephen Spackman}, {\it Steve VanDevender},
and {\it Paul Winner}, ported {\it NetHack\/} 3.1 to the PC.

%.pg
\medskip
\nd {\it Jon W\{tte} and {\it Hao-yang Wang},
with help from {\it Ross Brown}, {\it Mike Engber}, {\it David Hairston},
{\it Michael Hamel}, {\it Jonathan Handler}, {\it Johnny Lee},
{\it Tim Lennan}, {\it Rob Menke}, and {\it Andy Swanson},
developed {\it NetHack\/} 3.1 for the Macintosh, porting it for MPW.
Building on their development, {\it Barton House} added a Think C port.

%.pg
\medskip
\nd {\it Timo Hakulinen\/} ported {\it NetHack\/} 3.1 to OS/2.
{\it Eric Smith\/} ported {\it NetHack\/} 3.1 to the Atari.
{\it Pat Rankin}, with help from {\it Joshua Delahunty},
was responsible for the VMS version of {\it NetHack\/} 3.1.
{\it Michael Allison} ported {\it NetHack\/} 3.1 to Windows NT.

%.pg
\medskip
\nd {\it Dean Luick}, with help from {\it David Cohrs}, developed {\it NetHack\/}
3.1 for X11.
{\it Warwick Allison} wrote a tiled version of NetHack for the Atari;
he later contributed the tiles to the DevTeam and tile support was
then added to other platforms.

%.pg
\medskip
\nd The 3.2 development team, comprised of {\it Michael Allison}, {\it Ken
Arromdee}, {\it David Cohrs}, {\it Jessie Collet}, {\it Steve Creps}, {\it
Kevin Darcy}, {\it Timo Hakulinen}, {\it Steve Linhart}, {\it Dean Luick},
{\it Pat Rankin}, {\it Eric Smith}, {\it Mike Stephenson}, {\it Janet Walz},
and {\it Paul Winner}, released version 3.2 in April of 1996.

%.pg
\medskip
\nd Version 3.2 marked the tenth anniversary of the formation of the development
team.  In a testament to their dedication to the game, all thirteen members
of the original development team remained on the team at the start of work on
that release.  During the interval between the release of 3.1.3 and 3.2,
one of the founding members of the development team, {\it Dr. Izchak Miller},
was diagnosed with cancer and passed away.  That release of the game was
dedicated to him by the development and porting teams.

%.pg
\medskip
During the lifespan of {\it NetHack\/} 3.1 and 3.2, several enthusiasts
of the game added
their own modifications to the game and made these ``variants'' publicly
available:

%.pg
\medskip
{\it Tom Proudfoot} and {\it Yuval Oren} created {\it NetHack++},
which was quickly renamed {\it NetHack$--$}.
Working independently, {\it Stephen White} wrote {\it NetHack Plus}.
{\it Tom Proudfoot} later merged {\it NetHack Plus}
and his own {\it NetHack$--$} to produce {\it SLASH}.
{\it Larry Stewart-Zerba} and {\it Warwick Allison} improved the spell
casting system with the Wizard Patch.
{\it Warwick Allison} also ported NetHack to use the Qt interface.

%.pg
\medskip
{\it Warren Cheung} combined {\it SLASH} with the Wizard Patch
to produce {\it Slash'em\/}, and
with the help of {\it Kevin Hugo}, added more features.
Kevin later joined the
DevTeam and incorporated the best of these ideas into NetHack 3.3.

%.pg
\medskip
The final update to 3.2 was the bug fix release 3.2.3, which was released
simultaneously with 3.3.0 in December 1999 just in time for the Year 2000.

%.pg
\medskip
The 3.3 development team, consisting of {\it Michael Allison}, {\it Ken Arromdee}, 
{\it David Cohrs}, {\it Jessie Collet}, {\it Steve Creps}, {\it Kevin Darcy}, 
{\it Timo Hakulinen}, {\it Kevin Hugo}, {\it Steve Linhart}, {\it Ken Lorber}, 
{\it Dean Luick}, {\it Pat Rankin}, {\it Eric Smith}, {\it Mike Stephenson}, 
{\it Janet Walz}, and {\it Paul Winner}, released 3.3.0 in 
December 1999 and 3.3.1 in August of 2000.

%.pg
\medskip
Version 3.3 offered many firsts. It was the first version to separate race 
and profession. The Elf class was removed in preference to an elf race, 
and the races of dwarves, gnomes, and orcs made their first appearance in 
the game alongside the familiar human race.  Monk and Ranger roles joined 
Archeologists, Barbarians, Cavemen, Healers, Knights, Priests, Rogues, Samurai, 
Tourists, Valkyries and of course, Wizards.  It was also the first version
to allow you to ride a steed, and was the first version to have a publicly 
available web-site listing all the bugs that had been discovered.  Despite 
that constantly growing bug list, 3.3 proved stable enough to last for
more than a year and a half.

%.pg
\medskip
The 3.4 development team initially consisted of 
{\it Michael Allison}, {\it Ken Arromdee},
{\it David Cohrs}, {\it Jessie Collet}, {\it Kevin Hugo}, {\it Ken Lorber},
{\it Dean Luick}, {\it Pat Rankin}, {\it Mike Stephenson}, 
{\it Janet Walz}, and {\it Paul Winner}, with {\it  Warwick Allison} joining 
just before the release of NetHack 3.4.0 in March 2002.

%.pg
\medskip
As with version 3.3, various people contributed to the game as a whole as
well as supporting ports on the different platforms that {\it NetHack\/}
runs on:

%.pg
\medskip
\nd{\it Pat Rankin} maintained 3.4 for VMS.

%.pg
\medskip
\nd {\it Michael Allison} maintained NetHack 3.4 for the MS-DOS platform.
{\it Paul Winner} and {\it Yitzhak Sapir} provided encouragement.

%.pg
\medskip
\nd {\it Dean Luick}, {\it Mark Modrall}, and {\it Kevin Hugo} maintained and
enhanced the Macintosh port of 3.4.

%.pg
\medskip
\nd {\it Michael Allison}, {\it David Cohrs}, {\it Alex Kompel}, {\it Dion Nicolaas}, and 
{\it Yitzhak Sapir} maintained and enhanced 3.4 for the Microsoft Windows platform.
{\it Alex Kompel} contributed a new graphical interface for the Windows port. 
{\it Alex Kompel} also contributed a Windows CE port for 3.4.1.

%.pg
\medskip
\nd {\it Ron Van Iwaarden} maintained 3.4 for OS/2.

%.pg
\medskip
\nd {\it Janne Salmij\"{a}rvi} and {\it Teemu Suikki} maintained
and enhanced the Amiga port of 3.4 after {\it Janne Salmij\"{a}rvi} resurrected
it for 3.3.1.

%.pg
\medskip
\nd {\it Christian ``Marvin'' Bressler} maintained 3.4 for the Atari after he
resurrected it for 3.3.1.

%.pg
\medskip
\nd There is a NetHack web site maintained by {\it Ken Lorber} at 
http:{\tt /}{\tt /}www.nethack.org{\tt /}.

%.pg
\bigskip
\nd From time to time, some depraved individual out there in netland sends a
particularly intriguing modification to help out with the game.  The Gods of
the Dungeon sometimes make note of the names of the worst of these miscreants
in this, the list of Dungeoneers:

%.sd
\begin{center}
\begin{tabular}{lll}
%TABLE_START
Adam Aronow & Izchak Miller & Mike Stephenson\\
Alex Kompel & J. Ali Harlow & Norm Meluch\\
Andreas Dorn & Janet Walz & Olaf Seibert\\
Andy Church & Janne Salmij\"{a}rvi & Pasi Kallinen\\
Andy Swanson & Jean-Christophe Collet & Pat Rankin\\
Ari Huttunen & Jochen Erwied & Paul Winner\\
Barton House & John Kallen & Pierre Martineau\\
Benson I. Margulies & John Rupley & Ralf Brown\\
Bill Dyer & John S. Bien & Ray Chason\\
Boudewijn Waijers & Johnny Lee & Richard Addison\\
Bruce Cox & Jon W\{tte & Richard Beigel\\
Bruce Holloway & Jonathan Handler & Richard P. Hughey\\
Bruce Mewborne & Joshua Delahunty & Rob Menke\\
Carl Schelin & Keizo Yamamoto & Robin Johnson\\
Chris Russo & Ken Arnold & Roderick Schertler\\
David Cohrs & Ken Arromdee & Roland McGrath\\
David Damerell & Ken Lorber & Ron Van Iwaarden\\
David Gentzel & Ken Washikita & Ronnen Miller\\
David Hairston & Kevin Darcy & Ross Brown\\
Dean Luick & Kevin Hugo & Sascha Wostmann\\
Del Lamb & Kevin Sitze & Scott Bigham\\
Deron Meranda & Kevin Smolkowski & Scott R. Turner\\
Dion Nicolaas & Kevin Sweet & Stephen Spackman\\
Dylan O'Donnell & Lars Huttar & Stephen White\\
Eric Backus & Malcolm Ryan & Steve Creps\\
Eric Hendrickson & Mark Gooderum & Steve Linhart\\
Eric R. Smith & Mark Modrall & Steve VanDevender\\
Eric S. Raymond & Marvin Bressler & Teemu Suikki\\
Erik Andersen & Matthew Day & Tim Lennan\\
Frederick Roeber & Merlyn LeRoy & Timo Hakulinen\\
Gil Neiger & Michael Allison & Tom Almy\\
Greg Laskin & Michael Feir & Tom West\\
Greg Olson & Michael Hamel & Warren Cheung\\
Gregg Wonderly & Michael Sokolov & Warwick Allison\\
Hao-yang Wang & Mike Engber & Yitzhak Sapir\\
Helge Hafting & Mike Gallop\\
Irina Rempt-Drijfhout & Mike Passaretti
%TABLE_END  Do not delete this line.
\end{tabular}
\end{center}
%.ed

%\vfill
%\begin{flushleft}
%\small
%Microsoft and MS-DOS are registered trademarks of Microsoft Corporation.\\
%%%Don't need next line if a UNIX macro automatically inserts footnotes.
%UNIX is a registered trademark of AT\&T.\\
%Lattice is a trademark of Lattice, Inc.\\
%Atari and 1040ST are trademarks of Atari, Inc.\\
%AMIGA is a trademark of Commodore-Amiga, Inc.\\
%%.sm
%Brand and product names are trademarks or registered trademarks
%of their respective holders.
%\end{flushleft}

\end{document}
